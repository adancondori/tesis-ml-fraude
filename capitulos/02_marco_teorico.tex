% CAPÍTULO 1: REFERENTES TEÓRICOS
\chapter{Referentes Teóricos}

\section*{Referencia a núcleos teóricos a desarrollar en la investigación}

\subsection*{Objeto de estudio}

El objeto de estudio de esta investigación es el diseño e implementación de un modelo de aprendizaje automático supervisado que permita detectar anomalías y fraudes en transacciones electrónicas, en el contexto de la empresa TechSport. Esta plataforma SaaS procesa pagos a través de múltiples pasarelas y canales digitales, lo que genera una alta exposición a riesgos operacionales y económicos por posibles fraudes no detectados a tiempo. El estudio se enfoca en la aplicación de algoritmos de clasificación para el análisis de datos históricos de transacciones.

\subsection*{Campo de acción}

El campo de acción se enmarca en el sector fintech, específicamente en el análisis de pagos digitales y la seguridad transaccional en plataformas tecnológicas. La empresa objeto de estudio, TechSport, opera en múltiples países y administra pagos de reservas deportivas mediante canales móviles, web y físicos. Esta investigación se desarrolla en un entorno técnico caracterizado por arquitecturas distribuidas, diversidad de pasarelas de pago y necesidad de cumplimiento normativo en materia de protección de datos y prevención del fraude financiero.

\subsection*{Fundamento teórico del objetivo general}

El objetivo general de esta investigación se sustenta en tres ejes teóricos: (1) los principios del aprendizaje automático supervisado, que permiten entrenar modelos con datos etiquetados para clasificar nuevas observaciones; (2) los enfoques de detección de fraude financiero, que buscan identificar patrones atípicos en los datos transaccionales; y (3) los fundamentos de seguridad en sistemas digitales de pago, que exigen soluciones capaces de proteger la integridad y confiabilidad de las operaciones. Estos núcleos teóricos permiten orientar el diseño del modelo propuesto hacia una solución técnica viable y alineada con los requerimientos actuales del entorno fintech.

\subsection*{Índice Tentativo para el Desarrollo del Marco Teórico}

\subsubsection*{Breve estado del arte sobre la detección de fraudes en pagos electrónicos}

1.1.1. Panorama global del fraude financiero digital

1.1.2. Casos relevantes y estadísticas recientes en plataformas fintech

1.1.3. Estudios previos sobre fraude en entornos SaaS y multicanal

\subsubsection*{Fundamentos teóricos del aprendizaje automático}

1.2.1. Definición y evolución del aprendizaje automático

1.2.2. Tipos de aprendizaje automático: supervisado, no supervisado y por refuerzo

1.2.3. Modelos supervisados aplicables a detección de fraude

1.2.4. Métricas de evaluación en modelos de clasificación: precisión, recall, F1-score

\subsubsection*{Teorías y enfoques en la detección de anomalías y fraude}

1.3.1. Concepto de anomalía en datos transaccionales

1.3.2. Técnicas estadísticas vs. técnicas basadas en IA

1.3.3. Enfoques híbridos en la detección de fraude

1.3.4. Limitaciones de los sistemas basados en reglas estáticas

\subsubsection*{Seguridad digital y gestión de riesgo en pagos electrónicos}

1.4.1. Principios de seguridad en sistemas de pago (Confidencialidad, Integridad y Disponibilidad - CIA)

1.4.2. Normativas internacionales: PCI DSS, AML/KYC, GDPR

1.4.3. Gestión del riesgo operativo y transaccional en entornos digitales

1.4.4. Recomendaciones para plataformas con múltiples pasarelas de pago

\subsubsection*{Aplicación del aprendizaje automático en entornos fintech}

1.5.1. Uso de Machine Learning en sistemas de pago y comercio electrónico

1.5.2. Plataformas SaaS y su exposición al fraude

1.5.3. Estudios de caso: modelos aplicados a detección de fraude con tarjetas, wallets y otros medios

1.5.4. Evaluación de impacto y beneficios de la IA en la reducción de fraudes

\subsubsection*{Bases técnicas para el desarrollo del modelo propuesto}

1.6.1. Preparación y limpieza de datos históricos de transacciones

1.6.2. Selección del algoritmo y criterios de entrenamiento

1.6.3. División del conjunto de datos: entrenamiento, validación y prueba

1.6.4. Herramientas y librerías utilizadas (Scikit-learn, TensorFlow, Pandas, etc.)
