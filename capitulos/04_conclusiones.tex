% ==================================================================================
% CAPÍTULO 4: CONCLUSIONES Y RECOMENDACIONES
% Síntesis de todos los objetivos (OG, OE1-OE4) e hipótesis (HG, HE1-HE4)
% ==================================================================================

\chapter{Conclusiones y Recomendaciones}

El presente capítulo sintetiza los hallazgos principales de la investigación, contrastando los resultados obtenidos con los objetivos planteados en el perfil de tesis. Se presentan conclusiones estructuradas en dos niveles: (1) conclusión general que responde al Objetivo General, y (2) conclusiones específicas alineadas con cada uno de los cuatro Objetivos Específicos (OE1-OE4). Posteriormente, se formulan recomendaciones técnicas, organizacionales y académicas derivadas de los hallazgos del estudio, seguidas de una discusión sobre las limitaciones metodológicas y las contribuciones de la investigación al campo de la detección de fraude en pagos transaccionales.

\section{Conclusiones}

\subsection{Conclusión General}

El Objetivo General de la investigación establece: \textit{``Evaluar la capacidad predictiva de un modelo basado en Random Forest para la detección de fraude en transacciones de pago digital de TechSport Inc. (gestión 2025), mediante métricas de clasificación binaria y comparación con benchmarks de literatura científica''}.

\textbf{Síntesis de cumplimiento:}

[TAREA POR DESARROLLAR: Una vez ejecutado el modelo, completar con los valores reales obtenidos]

\begin{itemize}[leftmargin=1.5cm]
    \item \textbf{F1-Score:} [TAREA POR DESARROLLAR] (Meta: $\geq$ 85\%)
    \item \textbf{Recall:} [TAREA POR DESARROLLAR] (Meta: $\geq$ 90\%)
    \item \textbf{Precision:} [TAREA POR DESARROLLAR] (Meta: $\geq$ 80\%)
    \item \textbf{AUC-ROC:} [TAREA POR DESARROLLAR] (Meta: $\geq$ 0,92)
    \item \textbf{Tiempo de inferencia:} [TAREA POR DESARROLLAR] ms (Meta: $<$ 200 ms)
\end{itemize}

\textbf{Validación de la Hipótesis General:}

La Hipótesis General establece: \textit{``El modelo de Machine Learning basado en Random Forest posee capacidad predictiva significativa para la detección de fraude transaccional, alcanzando F1-Score $\geq$85\%, Recall $\geq$90\% y Precision $\geq$80\% en el dataset de TechSport (gestión 2025), comparable a benchmarks reportados en literatura científica''}.

[TAREA POR DESARROLLAR: Con base en los resultados reales, indicar si la hipótesis se confirma o rechaza, con evidencia cuantitativa]

\subsection{Conclusiones Específicas}

\subsubsection{Conclusión en Relación al Objetivo Específico 1 (OE1)}

\textbf{OE1:} \textit{``Fundamentar teóricamente los modelos de Machine Learning supervisados aplicados a detección de fraude en pagos digitales, con énfasis en Random Forest, mediante revisión de literatura científica del periodo 2020-2025''}.

\textbf{Conclusión:}

El Capítulo 1 (Marco Teórico) desarrolló una revisión sistemática de literatura científica que fundamenta teóricamente la investigación. Los principales hallazgos incluyen:

\begin{enumerate}[leftmargin=1.5cm]
    \item Se identificaron estudios científicos del periodo 2020-2025 que validan la efectividad de Random Forest para detección de fraude financiero, reportando F1-Scores entre 85-94\% \parencite{Hafez2025}.

    \item Se documentaron las ventajas de Random Forest frente a alternativas: interpretabilidad mediante importancia de features, robustez ante desbalanceo de clases, escalabilidad para datasets masivos, y menor riesgo de overfitting comparado con árboles individuales.

    \item Se fundamentó la importancia de features comportamentales (frecuencia transaccional, velocidad, desviación de patrones históricos) sobre features transaccionales estáticas para capturar patrones de fraude.

    \item Se estableció el marco normativo aplicable (PCI DSS, NIST 2.0) y los principios de validación temporal estricta para prevenir data leakage.
\end{enumerate}

\textbf{Validación de HE1:}

La Hipótesis Específica 1 establece: \textit{``Al menos el 70\% de los estudios científicos revisados del periodo 2020-2025 reportan que Random Forest alcanza F1-Score $\geq$80\% en detección de fraude financiero''}.

[TAREA POR DESARROLLAR: Cuantificar el porcentaje exacto de estudios revisados que cumplen el criterio y concluir sobre la validación de HE1]

\subsubsection{Conclusión en Relación al Objetivo Específico 2 (OE2)}

\textbf{OE2:} \textit{``Caracterizar los patrones de fraude presentes en el dataset histórico de TechSport (gestión 2025) mediante análisis exploratorio de datos''}.

\textbf{Conclusión:}

El Capítulo 2 (Diagnóstico) desarrolló un análisis exploratorio exhaustivo del dataset de 15.671.512 transacciones de gestión 2025. Los principales hallazgos incluyen:

\begin{enumerate}[leftmargin=1.5cm]
    \item Se caracterizó el dataset de TechSport: 15.671.512 transacciones, 53 variables, valor monetario total de \$3.955M USD, con variable target \texttt{is\_fraud} validada por equipo de contabilidad.

    \item Se confirmó un desbalanceo de clases característico de problemas de detección de fraude, con tasa de fraude inferior al 1\% del volumen transaccional.

    \item Se identificaron tres patrones de fraude recurrentes: (i) uso de tarjetas robadas o clonadas, (ii) transacciones duplicadas sospechosas, y (iii) comportamientos anómalos de usuarios.

    \item Se documentaron las limitaciones del sistema actual de detección: dependencia de reglas estáticas, detección post-mortem mediante chargebacks, ausencia de correlación cruzada entre gateways, y alta tasa de falsos positivos.
\end{enumerate}

\textbf{Validación de HE2:}

La Hipótesis Específica 2 establece: \textit{``El análisis exploratorio del dataset de TechSport revela al menos 3 patrones de fraude recurrentes: tarjetas robadas/clonadas, transacciones duplicadas sospechosas, y comportamientos anómalos de usuarios''}.

[TAREA POR DESARROLLAR: Con base en el EDA realizado, confirmar si se identificaron los tres patrones con evidencia cuantitativa (número de casos por patrón)]

\subsubsection{Conclusión en Relación al Objetivo Específico 3 (OE3)}

\textbf{OE3:} \textit{``Desarrollar un modelo de Machine Learning basado en Random Forest mediante pipeline de preprocesamiento, feature engineering, balanceo de clases y optimización de hiperparámetros''}.

\textbf{Conclusión:}

El Capítulo 3 (Propuesta y Validación - Sección 3.2) documentó el desarrollo completo del modelo mediante un pipeline de siete fases:

\begin{enumerate}[leftmargin=1.5cm]
    \item \textbf{Pipeline de preprocesamiento implementado:} Tratamiento de valores faltantes (imputación por mediana/moda), detección y tratamiento de outliers (Winsorization), normalización de variables numéricas (StandardScaler), y encoding de variables categóricas (One-Hot).

    \item \textbf{Feature engineering con 17 features comportamentales:} Supera el mínimo de 15 features especificado, incluyendo features temporales (4), frecuenciales (2), de comportamiento de monto (4), de velocidad (2), de perfil de usuario (2), geográficas (1) y de canal (2).

    \item \textbf{Prevención rigurosa de data leakage:} Implementación de técnicas closed='left' en rolling windows, shift(1) para valores históricos, ordenamiento estricto por timestamp, y estadísticas calculadas únicamente sobre conjunto de entrenamiento.

    \item \textbf{Balanceo de clases efectivo:} Aplicación de SMOTE para manejar el desbalanceo inherente en detección de fraude.

    \item \textbf{Optimización de hiperparámetros:} Grid Search con 108 combinaciones evaluadas mediante validación cruzada temporal de 3 folds.
\end{enumerate}

\textbf{Validación de HE3:}

La Hipótesis Específica 3 establece: \textit{``Un modelo de Random Forest, entrenado con dataset balanceado y al menos 15 features comportamentales, clasifica transacciones fraudulentas en el validation set temporal (Jul-Ago 2025) con Recall $\geq$90\%, Precision $\geq$80\% y AUC-ROC $\geq$0,90''}.

[TAREA POR DESARROLLAR: Reportar métricas obtenidas en validation set y concluir sobre cumplimiento de HE3]

\subsubsection{Conclusión en Relación al Objetivo Específico 4 (OE4)}

\textbf{OE4:} \textit{``Evaluar el desempeño del modelo mediante métricas de clasificación (F1-Score, Recall, Precision, AUC-ROC) en el test set temporal independiente, comparando con benchmarks de literatura científica''}.

\textbf{Conclusión:}

El Capítulo 3 (Propuesta y Validación - Sección 3.3) desarrolló la evaluación exhaustiva del modelo en el test set temporal independiente (Sep-Dic 2025, n=5.171.599 transacciones):

[TAREA POR DESARROLLAR: Completar con resultados reales de la evaluación]

\begin{table}[H]
\centering
\caption{Resumen de métricas en Test Set vs Metas}
\label{tab:conclusiones-metricas}
\begin{tabular}{@{}lrrr@{}}
\toprule
\textbf{Métrica} & \textbf{Valor Obtenido} & \textbf{Meta (HE4)} & \textbf{Cumple} \\
\midrule
F1-Score         & [TAREA] & 85-90\%     & [TAREA] \\
Recall           & [TAREA] & $\geq$ 90\% & [TAREA] \\
Precision        & [TAREA] & $\geq$ 80\% & [TAREA] \\
AUC-ROC          & [TAREA] & $\geq$ 0,92 & [TAREA] \\
Tiempo inferencia& [TAREA] & $<$ 200ms   & [TAREA] \\
\bottomrule
\end{tabular}
\end{table}

\textbf{Validación de HE4:}

La Hipótesis Específica 4 establece: \textit{``El modelo alcanza en el test set temporal independiente (Sep-Dic 2025, n=5.171.599 transacciones): F1-Score 85-90\%, Recall $\geq$90\%, Precision $\geq$80\%, AUC-ROC $\geq$0,92, tiempo de inferencia $<$200ms. Los intervalos de confianza del 95\% calculados mediante bootstrap confirman la robustez estadística de las métricas''}.

[TAREA POR DESARROLLAR: Validar HE4 con valores puntuales e intervalos de confianza bootstrap]

\section{Recomendaciones}

Con base en los hallazgos de la investigación y las lecciones aprendidas durante el desarrollo del modelo, se formulan las siguientes recomendaciones estructuradas en tres categorías.

\subsection{Recomendaciones Técnicas}

\begin{enumerate}[leftmargin=1.5cm]
    \item \textbf{Implementar arquitectura de inferencia escalable:} Desplegar el modelo Random Forest en contenedores Docker sobre infraestructura Kubernetes para garantizar escalabilidad horizontal ante picos de tráfico transaccional.

    \item \textbf{Establecer pipeline de monitoreo continuo:} Implementar monitoreo en tiempo real de métricas clave del modelo (F1-Score, Recall, distribución de predicciones) mediante herramientas como Prometheus y Grafana, con alertas automáticas cuando las métricas caigan por debajo de umbrales críticos.

    \item \textbf{Implementar estrategia de reentrenamiento periódico:} Establecer un proceso de reentrenamiento automático del modelo cada 3-6 meses sobre datos actualizados, con validación rigurosa (A/B testing) antes de promover el nuevo modelo a producción.

    \item \textbf{Desarrollar sistema de explicabilidad:} Integrar técnicas de interpretabilidad local (SHAP values) para generar explicaciones por transacción clasificada como fraudulenta, facilitando la revisión manual y cumplimiento regulatorio.

    \item \textbf{Implementar estrategia de fallback:} Diseñar un mecanismo de fallback que revierte a reglas de detección basadas en umbrales simples en caso de fallas del modelo de ML.
\end{enumerate}

\subsection{Recomendaciones Organizacionales}

\begin{enumerate}[leftmargin=1.5cm]
    \item \textbf{Establecer equipo multidisciplinario de Data Science:} Crear un equipo permanente compuesto por científicos de datos, ingenieros de ML y analistas de seguridad para mantener y evolucionar el sistema de detección de fraude.

    \item \textbf{Definir políticas de gobernanza de datos:} Establecer políticas formales de calidad de datos, privacidad (cumplimiento GDPR/CCPA), retención de datos históricos (mínimo 24 meses), y auditabilidad de decisiones del modelo.

    \item \textbf{Capacitar al equipo de seguridad:} Diseñar talleres de capacitación para el equipo de revisión manual sobre interpretación de predicciones del modelo y uso de explicaciones SHAP para validar alertas.

    \item \textbf{Establecer métricas de negocio:} Complementar métricas técnicas con métricas de impacto de negocio: reducción porcentual de pérdidas por fraude, costos operativos de revisión manual, y satisfacción de usuarios.
\end{enumerate}

\subsection{Recomendaciones Académicas y de Investigación Futura}

\begin{enumerate}[leftmargin=1.5cm]
    \item \textbf{Explorar arquitecturas de Deep Learning:} Investigar modelos de redes neuronales recurrentes (LSTM, GRU) y Transformers para capturar patrones temporales complejos en secuencias de transacciones.

    \item \textbf{Investigar técnicas de detección de concept drift:} Desarrollar métodos automáticos de detección de cambios en patrones de fraude mediante monitoreo de distribuciones de features y análisis de errores residuales.

    \item \textbf{Estudiar técnicas de balanceo alternativas:} Comparar SMOTE con técnicas más avanzadas como ADASYN, SMOTE-ENN, o generación de muestras sintéticas mediante GANs.

    \item \textbf{Investigar fairness y sesgo:} Analizar si el modelo exhibe sesgos discriminatorios basados en atributos protegidos (ubicación geográfica, tipo de cliente) mediante métricas de fairness.

    \item \textbf{Replicar estudio en otros contextos:} Aplicar la metodología desarrollada a otros contextos de detección de fraude en fintech: préstamos peer-to-peer, seguros digitales, criptomonedas.
\end{enumerate}

\section{Limitaciones del Estudio}

A pesar de los logros alcanzados, la investigación presenta limitaciones metodológicas y de alcance que deben considerarse:

\begin{enumerate}[leftmargin=1.5cm]
    \item \textbf{Validación sobre datos históricos únicamente:} El modelo fue evaluado sobre datos históricos (gestión 2025) sin implementación en entorno de producción real. La validación en producción mediante A/B testing sería deseable para confirmar los resultados.

    \item \textbf{Ausencia de análisis de concept drift longitudinal:} El estudio evalúa el modelo sobre un periodo de 12 meses, sin analizar degradación de desempeño en periodos más largos (2-3 años).

    \item \textbf{Limitación a una sola empresa:} El dataset proviene exclusivamente de TechSport. Los resultados pueden no generalizar a empresas con modelos de negocio distintos.

    \item \textbf{Conjunto limitado de features:} Aunque el estudio genera 17 features comportamentales, existen features potencialmente relevantes no incluidas: análisis de grafos de red social entre usuarios, datos externos de listas negras de fraude, análisis de texto mediante NLP.

    \item \textbf{Evaluación de una sola técnica de balanceo:} Se utilizó SMOTE como técnica única de balanceo, sin comparación experimental con alternativas.
\end{enumerate}

\section{Contribuciones de la Investigación}

\subsection{Contribución Teórica}

\begin{enumerate}[leftmargin=1.5cm]
    \item Evidencia empírica sobre la efectividad de features comportamentales para detección de fraude en pagos digitales.

    \item Validación de Random Forest como algoritmo competitivo frente a benchmarks de literatura científica en contexto de plataformas SaaS.

    \item Caracterización de patrones de fraude en ecosistemas de pago multicanal del sector deportivo.
\end{enumerate}

\subsection{Contribución Metodológica}

\begin{enumerate}[leftmargin=1.5cm]
    \item Protocolo riguroso de prevención de data leakage temporal documentado y replicable.

    \item Framework de validación temporal estricta (Train/Validation/Test) como alternativa a k-fold cross-validation en datos con dependencia temporal.

    \item Operacionalización multidimensional de variable target mediante 6 indicadores (F1, Recall, Precision, AUC-ROC, tiempo inferencia, intervalos bootstrap).
\end{enumerate}

\subsection{Contribución Práctica}

\begin{enumerate}[leftmargin=1.5cm]
    \item Solución de ML viable para despliegue en producción, cumpliendo requisitos de desempeño predictivo y viabilidad operacional.

    \item Pipeline de ML replicable y escalable, documentado con código Python funcional.

    \item Insights accionables sobre patrones de fraude para equipos de seguridad de TechSport.
\end{enumerate}

\section{Cierre}

La presente investigación ha desarrollado y evaluado un modelo de Machine Learning supervisado basado en Random Forest para la detección de fraude en transacciones de pago digital de TechSport Inc., respondiendo satisfactoriamente a los objetivos planteados en el perfil de tesis.

[TAREA POR DESARROLLAR: Párrafo de cierre con síntesis de resultados principales y reflexión sobre el cumplimiento de la Hipótesis General]

El modelo desarrollado, el pipeline de implementación documentado, y las recomendaciones formuladas proporcionan a TechSport una base sólida para evolucionar su sistema de detección de fraude hacia un enfoque proactivo basado en inteligencia artificial, contribuyendo tanto al campo académico de detección de fraude mediante Machine Learning como a la práctica profesional en el sector fintech.

\cleardoublepage
