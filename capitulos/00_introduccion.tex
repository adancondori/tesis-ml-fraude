% ==================================================================================
% INTRODUCCIÓN
% ==================================================================================

\chapter*{Introducción}
\addcontentsline{toc}{chapter}{Introducción}
\markboth{INTRODUCCIÓN}{INTRODUCCIÓN}

La detección de fraude en los pagos digitales representa uno de los desafíos más críticos en la economía digital contemporánea, donde las transacciones electrónicas experimentan un crecimiento exponencial y las técnicas fraudulentas evolucionan constantemente. Según \textcite{Bello2024}, la detección de fraude en sistemas de pago requiere técnicas de inteligencia artificial mejoradas que puedan adaptarse y aprender de nuevos datos, mejorando su precisión y efectividad a lo largo del tiempo. Esta detección no solo es fundamental para proteger los activos financieros, sino también para preservar la confianza y la integridad de los ecosistemas digitales de pago.

A nivel global, las pérdidas por fraude en pagos digitales alcanzan cifras alarmantes. Según \textcite{HernandezAros2024}, el crecimiento exponencial de las transacciones digitales ha generado un aumento proporcional en las actividades fraudulentas, requiriendo sistemas de detección más sofisticados. En América Latina, esta problemática se intensifica debido a la rápida adopción de pagos digitales sin el correspondiente fortalecimiento de los sistemas de seguridad, donde las regiones emergentes enfrentan desafíos únicos relacionados con la diversidad de métodos de pago y patrones de comportamiento del consumidor.

En el contexto de los Estados Unidos, y más específicamente en Miami, Florida, la empresa TechSport —dedicada a la gestión y reserva digital de espacios deportivos— enfrenta desafíos operativos relacionados con la identificación de actividades fraudulentas en su sistema de pagos. Durante el periodo 2024-2025, se registraron intentos de fraude que no fueron detectados oportunamente por el sistema actual basado en reglas estáticas. Esta vulnerabilidad no solo expone a la empresa a pérdidas económicas, sino que también afecta la confianza del usuario, un intangible crítico para la sostenibilidad de las plataformas digitales.

Este panorama pone de manifiesto la necesidad de mejorar los sistemas de detección de fraude mediante el uso de técnicas avanzadas, como los modelos de aprendizaje automático (Machine Learning). Diversos estudios académicos han demostrado la efectividad de estos modelos, superando las limitaciones de los sistemas tradicionales. Por ejemplo, \textcite{Hafez2025} evidencian que los modelos supervisados alcanzan una precisión del 94.3\% en la identificación de fraudes con tarjetas de crédito, manteniendo una tasa baja de falsos positivos. Este enfoque algorítmico representa una solución prometedora en contextos donde los volúmenes de datos son elevados y las amenazas se transforman dinámicamente.

En el contexto regulatorio, \textcite{NIST2024} publicó en 2024 la versión 2.0 del Marco de Ciberseguridad del NIST (CSF 2.0), que incluye una nueva función denominada ``Govern'', enfatizando que la ciberseguridad es una fuente importante de riesgo empresarial. Esta actualización proporciona orientación específica para organizaciones de todos los tamaños, incluyendo sistemas de pago críticos que requieren protección contra amenazas avanzadas.

Los sistemas inteligentes aplicados a la detección de fraude representan una evolución natural en la protección de transacciones financieras digitales. La capacidad de procesar grandes volúmenes de datos transaccionales, identificar correlaciones no evidentes y generar alertas en tiempo real constituye el núcleo de los sistemas cognitivos modernos. Este enfoque supera las limitaciones de los métodos tradicionales basados en reglas estáticas, incorporando capacidades de aprendizaje adaptativo que mejoran continuamente la precisión de detección.

El presente trabajo de investigación se alinea directamente con el Área 1.2 Sistemas Inteligentes del documento regulatorio de la Unidad de Postgrado en Ciencias de la Computación y Telecomunicaciones de la Universidad Autónoma Gabriel René Moreno. Específicamente, este tema se enmarca dentro de la línea de investigación de Sistemas Cognitivos, abordando el desarrollo de sistemas capaces de reconocer patrones complejos y tomar decisiones automatizadas en entornos de alta concurrencia.

La presente investigación tiene como objetivo implementar un modelo de Machine Learning supervisado para la detección de anomalías y fraude en pagos digitales, utilizando un conjunto de datos históricos proporcionado por la empresa TechSport, ubicada en Miami, Florida, durante la gestión 2024-2025. El estudio es de tipo cuantitativo, aplicado, descriptivo-correlacional, con un diseño experimental en entorno controlado. Se evaluarán métricas clave como precisión, recall y F1-score, comparando el desempeño del modelo propuesto frente al sistema actual basado en reglas.

\cleardoublepage
