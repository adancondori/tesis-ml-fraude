% ==================================================================================
% PERFIL DE TESIS - ALINEADO CON METODOLOGÍA SAMPIERI (2018) Y MÉTODO AQP/CCA
% Maestría en Ciencias de la Computación - UAGRM - Gestión 2025
% Redacción impersonal según Sampieri
% ==================================================================================
% Documento base: Knowledge-Base-Rosario-Book/Tesis-Metodo-Cuadrados-Deteccion-Fraude.md
% ==================================================================================

% ==================================================================================
% 1. ANTECEDENTES DEL PROBLEMA
% Método AQP: Adónde (A), Quiénes (Q), Problema (P)
% Método CCA: Causas (C), Consecuencias (C), Aporte (A)
% ==================================================================================

\section*{1. Antecedentes del Problema}
\addcontentsline{toc}{section}{1. Antecedentes del Problema}

% ---------------------------------------------------------------------------
% 1.1 CONTEXTO GLOBAL DEL PROBLEMA
% ---------------------------------------------------------------------------

El fraude transaccional en pagos digitales constituye uno de los desafíos más críticos para la economía digital contemporánea. El crecimiento exponencial de las transacciones electrónicas, acompañado por la evolución constante de técnicas fraudulentas cada vez más sofisticadas, demanda sistemas de protección capaces de adaptarse dinámicamente a nuevas amenazas. Según \textcite{HernandezAros2024}, los sistemas de detección basados en reglas estáticas han quedado obsoletos, dado que los ataques actuales son dinámicos, adaptativos y evolucionan más rápidamente que la capacidad de actualización manual de reglas.

La literatura científica reciente evidencia que los modelos de Machine Learning supervisados ofrecen ventajas significativas en este contexto. \textcite{Hafez2025} demuestran, mediante una revisión sistemática de la literatura, que algoritmos como Random Forest y enfoques de ensemble learning alcanzan F1-Scores entre 85\% y 94\% en la detección de fraudes con tarjetas de crédito, superando sustancialmente el desempeño de sistemas basados en reglas predefinidas en términos de adaptabilidad, precisión y escalabilidad.

% ---------------------------------------------------------------------------
% 1.2 CONTEXTO REGIONAL
% ---------------------------------------------------------------------------

A nivel regional, esta problemática presenta características diferenciadas. En América Latina, la rápida adopción de tecnologías digitales ha incrementado significativamente la exposición a fraudes financieros, sin que ello haya estado acompañado por un desarrollo equivalente en mecanismos de prevención y detección. \textcite{OEABID2020} documentan brechas críticas en capacidades de monitoreo, análisis de amenazas y respuesta operativa en la región. La fragmentación del ecosistema ---derivada de la diversidad de medios de pago, regulaciones dispares entre países y niveles disímiles de madurez tecnológica--- crea un entorno propicio para la aparición de fraudes que evolucionan más rápido que los controles existentes.

En Estados Unidos, a pesar de contar con marcos regulatorios avanzados y tecnologías más maduras, el volumen masivo de transacciones procesadas diariamente, la creciente sofisticación de los ataques cibernéticos y la dependencia persistente de sistemas basados en reglas estáticas limitan la capacidad de respuesta efectiva frente a amenazas emergentes. El Marco de Ciberseguridad del NIST versión 2.0 \parencite{NIST2024} enfatiza que la ciberseguridad constituye una fuente importante de riesgo empresarial, proporcionando orientación específica para la protección de sistemas de pago críticos mediante enfoques adaptativos.

% ---------------------------------------------------------------------------
% 1.3 DELIMITACIÓN ESPACIAL (A - ADÓNDE)
% ---------------------------------------------------------------------------

En este contexto se ubica la empresa \textbf{TechSport Inc.}, plataforma SaaS (Software as a Service) internacional especializada en la gestión integral de instalaciones deportivas de raqueta (tenis, pádel, pickleball, basketball). La compañía tiene su sede principal en Miami, Florida, Estados Unidos, con alcance operacional internacional en múltiples países de América y Europa.

TechSport opera con una arquitectura tecnológica multicanal (Web, App Móvil, POS) integrada con más de diez pasarelas de pago internacionales:

\begin{itemize}
    \item Stripe (pasarela principal)
    \item CardConnect
    \item Kushki (Latinoamérica)
    \item AzulPay (República Dominicana)
    \item RazorPay (India)
    \item BAC (Centroamérica)
    \item Otros gateways regionales
\end{itemize}

La infraestructura de datos se sustenta en una base de datos ClickHouse (esquema \texttt{TechSport\_db\_production}), procesando más de 15 millones de transacciones anuales.

% ---------------------------------------------------------------------------
% 1.4 UNIDAD DE ANÁLISIS (Q - QUIÉNES)
% ---------------------------------------------------------------------------

Según \textcite[p. 174]{Hernandez2018}, \textit{``las unidades de análisis son los elementos sobre los cuales se recolectarán los datos''}. En esta investigación, la \textbf{unidad de análisis} es la \textbf{transacción de pago digital}.

Se define operacionalmente una transacción como un evento único de pago procesado a través de cualquier pasarela de pago integrada en TechSport, que contiene:

\begin{itemize}
    \item Identificador único (\texttt{transaction\_id})
    \item Monto y moneda
    \item Timestamp (fecha y hora)
    \item Canal de origen (Web, App, POS)
    \item Gateway utilizado
    \item Usuario asociado (\texttt{user\_id})
    \item Resultado (aprobada, rechazada, fraudulenta)
    \item Etiqueta de fraude (\texttt{is\_fraud}: 0 o 1)
\end{itemize}

\textbf{Población de estudio:} Totalidad de transacciones de pago procesadas por TechSport durante la gestión 2025, correspondientes a \textbf{15.671.512 registros}.

\begin{table}[H]
\centering
\caption{Distribución de transacciones por canal}
\label{tab:distribucion_canal}
\begin{tabular}{lrr}
\hline
\rowcolor{headerblue}
\textcolor{white}{\textbf{Canal}} & \textcolor{white}{\textbf{Porcentaje}} & \textcolor{white}{\textbf{Transacciones}} \\
\hline
Web & 64,59\% & 10.122.305 \\
App Móvil & 12,83\% & 2.010.635 \\
Transferencia bancaria & 12,61\% & 1.976.198 \\
POS (Punto de venta) & 8,44\% & 1.322.656 \\
Terminal móvil & 0,87\% & 136.340 \\
Otros & 0,66\% & 103.378 \\
\hline
\rowcolor{lightgreen}
\textbf{Total} & \textbf{100\%} & \textbf{15.671.512} \\
\hline
\end{tabular}
\end{table}

% ---------------------------------------------------------------------------
% 1.5 CRITERIOS DE INCLUSIÓN Y EXCLUSIÓN
% ---------------------------------------------------------------------------

\textbf{Criterios de Inclusión:}
\begin{enumerate}
    \item Transacciones procesadas entre el 01 de enero y 31 de diciembre de 2025
    \item Transacciones con estado final definido (aprobada, rechazada o fraudulenta)
    \item Transacciones con etiqueta \texttt{is\_fraud} validada por el equipo de contabilidad
    \item Transacciones con campos mínimos requeridos completos (transaction\_id, monto, timestamp, user\_id, gateway)
    \item Transacciones procesadas a través de cualquiera de los gateways integrados en TechSport
\end{enumerate}

\textbf{Criterios de Exclusión:}
\begin{enumerate}
    \item Transacciones de prueba o sandbox (ambientes de desarrollo)
    \item Transacciones con estado pendiente o incompleto al cierre del período
    \item Transacciones con datos corruptos o inconsistentes
    \item Transacciones internas de la empresa (transferencias entre cuentas TechSport)
    \item Transacciones con monto igual a cero (cortesías, promociones 100\%)
    \item Transacciones duplicadas por error de sistema
\end{enumerate}

% ---------------------------------------------------------------------------
% 1.6 IDENTIFICACIÓN DEL PROBLEMA (P - PROBLEMA / VARIABLE MADRE)
% ---------------------------------------------------------------------------

TechSport enfrenta un problema de \textbf{fraude transaccional} en sus pagos digitales, caracterizado por cinco manifestaciones principales:

\begin{enumerate}
    \item \textbf{Detección tardía:} Los fraudes se identifican post-mortem mediante chargebacks, entre 0 y 5 meses después de la transacción original.

    \item \textbf{Sistema reactivo:} Dependencia de reglas estáticas sin capacidad de aprendizaje automático; las reglas requieren actualización manual constante y no se adaptan a nuevos patrones de fraude.

    \item \textbf{Alta tasa de falsos positivos:} Rechazos incorrectos de pagos legítimos que afectan la experiencia del usuario y generan pérdida de ingresos.

    \item \textbf{Arquitectura fragmentada:} Múltiples gateways operando de forma aislada sin correlación cruzada de comportamientos; cada pasarela procesa independientemente sin visión unificada de riesgo.

    \item \textbf{Ausencia de predicción:} No existe modelo predictivo que alerte sobre transacciones sospechosas antes de su aprobación.
\end{enumerate}

% ---------------------------------------------------------------------------
% 1.7 CAUSAS DEL PROBLEMA (C - CAUSAS)
% ---------------------------------------------------------------------------

Las causas del fraude transaccional en TechSport se organizan en tres niveles según su naturaleza:

\textbf{Causas Técnicas:}
\begin{enumerate}
    \item \textbf{Ausencia de arquitectura unificada} para gestión de riesgo transaccional: no existe correlación entre comportamientos de diferentes gateways; cada pasarela opera de forma aislada.

    \item \textbf{Dependencia de reglas estáticas} sin aprendizaje automático: las reglas requieren actualización manual constante y no se adaptan a nuevos patrones de fraude.

    \item \textbf{Carencia de gobernanza sobre integraciones API:} dificulta trazabilidad y análisis contextual; inconsistencias en formatos de datos entre gateways.
\end{enumerate}

\textbf{Causas Operativas:}
\begin{enumerate}
    \setcounter{enumi}{3}
    \item \textbf{Proceso de etiquetado post-mortem:} fraudes identificados 0-5 meses después por chargebacks; imposibilidad de prevención en tiempo real.

    \item \textbf{Fragmentación del ecosistema de pagos:} 10+ pasarelas con lógicas diferentes; múltiples monedas y regulaciones.
\end{enumerate}

\textbf{Causas Organizacionales:}
\begin{enumerate}
    \setcounter{enumi}{5}
    \item \textbf{Ausencia de equipo especializado en fraud analytics:} no existen científicos de datos dedicados a fraude; el equipo de contabilidad gestiona manualmente los casos.
\end{enumerate}

% ---------------------------------------------------------------------------
% 1.8 CONSECUENCIAS DEL PROBLEMA (C - CONSECUENCIAS)
% ---------------------------------------------------------------------------

Si el problema de fraude transaccional persiste sin solución, las consecuencias se manifiestan en tres niveles:

\textbf{Consecuencias Económicas:}
\begin{enumerate}
    \item Pérdidas financieras directas por fraudes consumados
    \item Costos de chargebacks y disputas con bancos emisores
    \item Multas regulatorias por incumplimiento de PCI DSS / NIST
    \item Incremento de primas en seguros de procesamiento
\end{enumerate}

\textbf{Consecuencias Operativas:}
\begin{enumerate}
    \setcounter{enumi}{4}
    \item Alta tasa de falsos positivos que rechaza pagos legítimos
    \item Carga operativa excesiva en equipos de soporte y contabilidad
    \item Incapacidad de escalar el sistema de detección
\end{enumerate}

\textbf{Consecuencias Estratégicas:}
\begin{enumerate}
    \setcounter{enumi}{7}
    \item Deterioro de la confianza de usuarios institucionales (clubes deportivos)
    \item Pérdida de competitividad frente a plataformas con IA
    \item Riesgo reputacional por brechas de seguridad
\end{enumerate}

% ---------------------------------------------------------------------------
% 1.9 APORTE DE LA INVESTIGACIÓN (A - APORTE)
% ---------------------------------------------------------------------------

El presente estudio aporta una evaluación de la capacidad predictiva de un modelo de Machine Learning supervisado basado en Random Forest para la detección de fraude transaccional.

El aporte incluye:

\begin{enumerate}
    \item \textbf{Pipeline de preprocesamiento:} manejo de valores faltantes y outliers, normalización de variables numéricas, codificación de variables categóricas.

    \item \textbf{Feature Engineering} (mínimo 15 características): monto normalizado, frecuencia transaccional del usuario, velocidad transaccional (tiempo entre transacciones), hora del día y día de la semana, ratio monto/promedio histórico del usuario, historial de chargebacks previos, canal y gateway utilizados, geolocalización IP.

    \item \textbf{Estrategia de balanceo de clases:} SMOTE (Synthetic Minority Over-sampling Technique) o \texttt{class\_weight='balanced'} en Random Forest.

    \item \textbf{Validación temporal estricta:}
    \begin{itemize}
        \item Train: Ene-Jun 2025 (50\%) --- 7.835.756 transacciones
        \item Validation: Jul-Ago 2025 (17\%) --- 2.664.157 transacciones
        \item Test: Sep-Dic 2025 (33\%) --- 5.171.599 transacciones
    \end{itemize}

    \item \textbf{Métricas objetivo:}
    \begin{itemize}
        \item F1-Score $\geq$ 85\%
        \item Recall $\geq$ 90\% (detectar fraudes reales)
        \item Precision $\geq$ 80\% (minimizar falsos positivos)
        \item AUC-ROC $\geq$ 0,92
        \item Tiempo de inferencia $<$ 200ms
    \end{itemize}
\end{enumerate}

Hasta donde se ha podido verificar mediante revisión documental y análisis institucional, no existen proyectos anteriores ni en ejecución en TechSport que propongan una solución basada en técnicas de Machine Learning para la detección de fraude en pagos transaccionales.

% ==================================================================================
% 2. FORMULACIÓN DEL PROBLEMA
% ==================================================================================

\section*{2. Formulación del Problema}
\addcontentsline{toc}{section}{2. Formulación del Problema}

La arquitectura tecnológica de pagos multicanal implementada actualmente en TechSport presenta limitaciones estructurales y técnicas que dificultan la detección oportuna de transacciones fraudulentas. Esta situación incrementa los riesgos operacionales y compromete tanto la seguridad de las transacciones como la experiencia del usuario.

\subsection*{2.1. Problema General}
\addcontentsline{toc}{subsection}{2.1. Problema General}

\begin{quote}
\textbf{¿Cuál es la capacidad predictiva de un modelo basado en Random Forest para la detección de fraude en transacciones de pago digital de TechSport Inc. durante la gestión 2025?}
\end{quote}

\subsection*{2.2. Problemas Específicos}
\addcontentsline{toc}{subsection}{2.2. Problemas Específicos}

\textbf{PE1 (Fundamentación teórica):}
\begin{quote}
¿Cuál es el fundamento teórico-técnico que respalda el uso de modelos de Machine Learning supervisados, particularmente Random Forest, para la detección de fraude en pagos digitales según la literatura científica 2020-2025?
\end{quote}

\textbf{PE2 (Diagnóstico):}
\begin{quote}
¿Cuáles son las características y patrones de fraude presentes en el dataset histórico de transacciones de TechSport (gestión 2025)?
\end{quote}

\textbf{PE3 (Desarrollo):}
\begin{quote}
¿Cómo estructurar un modelo de Machine Learning basado en Random Forest que clasifique transacciones fraudulentas mediante pipeline de preprocesamiento, feature engineering y optimización de hiperparámetros?
\end{quote}

\textbf{PE4 (Evaluación):}
\begin{quote}
¿Qué nivel de desempeño (F1-Score, Recall, Precision, AUC-ROC) alcanza el modelo en el test set temporal independiente, y cómo se compara con benchmarks de literatura científica?
\end{quote}

\subsection*{2.3. Objeto de Estudio}
\addcontentsline{toc}{subsection}{2.3. Objeto de Estudio}

Fraude transaccional en pagos digitales procesados por plataformas SaaS multicanal.

\subsection*{2.4. Campo de Acción}
\addcontentsline{toc}{subsection}{2.4. Campo de Acción}

Aplicación y evaluación de modelos de Machine Learning supervisados (Random Forest) para la detección de fraude en pagos transaccionales de la empresa TechSport durante la gestión 2025.

% ==================================================================================
% 3. OBJETIVOS DE LA INVESTIGACIÓN
% ==================================================================================

\section*{3. Objetivos de la Investigación}
\addcontentsline{toc}{section}{3. Objetivos de la Investigación}

\subsection*{3.1. Objetivo General}
\addcontentsline{toc}{subsection}{3.1. Objetivo General}

\begin{quote}
\textbf{Evaluar} la capacidad predictiva de un modelo basado en Random Forest para la detección de fraude en transacciones de pago digital de TechSport Inc. (gestión 2025), mediante métricas de clasificación binaria y comparación con benchmarks de literatura científica.
\end{quote}

\subsection*{3.2. Objetivos Específicos}
\addcontentsline{toc}{subsection}{3.2. Objetivos Específicos}

\begin{enumerate}
    \item \textbf{Fundamentar teóricamente} los modelos de Machine Learning supervisados aplicados a detección de fraude en pagos digitales, con énfasis en Random Forest, mediante revisión de literatura científica del periodo 2020-2025.

    \item \textbf{Caracterizar} los patrones de fraude presentes en el dataset histórico de TechSport (gestión 2025) mediante análisis exploratorio de datos.

    \item \textbf{Desarrollar} un modelo de Machine Learning basado en Random Forest mediante pipeline de preprocesamiento, feature engineering, balanceo de clases y optimización de hiperparámetros.

    \item \textbf{Evaluar} el desempeño del modelo mediante métricas de clasificación (F1-Score, Recall, Precision, AUC-ROC) en el test set temporal independiente, comparando con benchmarks de literatura científica.
\end{enumerate}

% ==================================================================================
% 4. JUSTIFICACIÓN DE LA INVESTIGACIÓN
% ==================================================================================

\section*{4. Justificación de la Investigación}
\addcontentsline{toc}{section}{4. Justificación de la Investigación}

\subsection*{4.1. Justificación Teórica}
\addcontentsline{toc}{subsection}{4.1. Justificación Teórica}

El estudio contribuye al cuerpo de conocimientos en \textbf{Machine Learning aplicado a seguridad financiera}, validando empíricamente la efectividad de Random Forest en un contexto real de pagos digitales multicanal. Los hallazgos aportan evidencia sobre la aplicabilidad de técnicas de ensemble learning en plataformas SaaS del sector deportivo, ampliando el alcance de la literatura existente que se concentra principalmente en banca tradicional y e-commerce.

\subsection*{4.2. Justificación Práctica}
\addcontentsline{toc}{subsection}{4.2. Justificación Práctica}

La investigación responde a una \textbf{necesidad operativa concreta} de TechSport, que requiere mejorar su capacidad de detección de fraude para reducir pérdidas económicas, disminuir falsos positivos, y cumplir con normativas internacionales (PCI DSS, NIST). El modelo desarrollado es transferible a organizaciones similares (SaaS multicanal deportivas o fintech).

\subsection*{4.3. Justificación Económica}
\addcontentsline{toc}{subsection}{4.3. Justificación Económica}

La detección efectiva de fraude \textbf{previene pérdidas financieras} directas (fraudes consumados) e indirectas (chargebacks, disputas, multas regulatorias). Un modelo con Recall $\geq$90\% implica detectar 90\% de fraudes que actualmente pasan desapercibidos, generando ROI positivo.

\textbf{Estimación de ahorro proyectado:}

\begin{table}[H]
\centering
\caption{Estimación de ahorro económico proyectado}
\label{tab:ahorro_economico}
\begin{tabular}{|p{5cm}|p{4cm}|p{4cm}|}
\hline
\rowcolor{headerblue}
\textcolor{white}{\textbf{Concepto}} & \textcolor{white}{\textbf{Cálculo}} & \textcolor{white}{\textbf{Estimación Anual}} \\
\hline
Transacciones totales & 15.671.512 & - \\
\hline
Tasa de fraude estimada & $\sim$0,5\% & $\sim$78.357 fraudes \\
\hline
Monto promedio por fraude & $\sim$\$150 USD & - \\
\hline
Pérdida potencial total & 78.357 $\times$ \$150 & $\sim$\$11.753.550 USD \\
\hline
Detección actual (estimada) & $\sim$40\% & $\sim$\$4.701.420 USD \\
\hline
Detección con modelo (Recall 90\%) & 90\% & $\sim$\$10.578.195 USD \\
\hline
\rowcolor{lightgreen}
\textbf{Ahorro incremental proyectado} & \textbf{90\% - 40\%} & \textbf{$\sim$\$5.876.775 USD/año} \\
\hline
\end{tabular}
\end{table}

\textit{Nota: Estimaciones basadas en benchmarks de la industria fintech. Los valores reales serán calculados con datos de TechSport durante el diagnóstico (OE2).}

\subsection*{4.4. Justificación Metodológica}
\addcontentsline{toc}{subsection}{4.4. Justificación Metodológica}

El estudio aplica \textbf{rigurosidad metodológica} según \textcite{Hernandez2018} en un contexto de ciencias computacionales, demostrando que las investigaciones de Machine Learning pueden estructurarse con el mismo rigor que investigaciones en ciencias sociales. El pipeline reproducible y la validación estadística (bootstrap con intervalos de confianza del 95\%) aportan un modelo metodológico replicable.

\subsection*{4.5. Justificación Social}
\addcontentsline{toc}{subsection}{4.5. Justificación Social}

La investigación protege a \textbf{usuarios finales} (atletas, clubes deportivos) de ser víctimas de fraude o de ver rechazados sus pagos legítimos. Contribuye a un ecosistema de pagos digitales más seguro y confiable.

\subsection*{4.6. Justificación Investigativa}
\addcontentsline{toc}{subsection}{4.6. Justificación Investigativa}

El estudio deja abierta la posibilidad de que otros investigadores amplíen los hallazgos, aplicando el modelo a otros contextos fintech o comparando con otros algoritmos de Machine Learning.

% ==================================================================================
% 5. FORMULACIÓN DE LA CONSTRUCCIÓN TEÓRICA. HIPÓTESIS PARA DEFENDER
% ==================================================================================

\section*{5. Hipótesis para Defender}
\addcontentsline{toc}{section}{5. Hipótesis para Defender}

Según \textcite[p. 107]{Hernandez2018}: \textit{``Las hipótesis son explicaciones tentativas del fenómeno investigado que se formulan como proposiciones''}. Para investigaciones correlacionales-explicativas, las hipótesis deben especificar la relación esperada entre variables.

\subsection*{5.1. Hipótesis General}
\addcontentsline{toc}{subsection}{5.1. Hipótesis General}

\begin{quote}
El modelo de Machine Learning basado en Random Forest posee capacidad predictiva significativa para la detección de fraude transaccional, alcanzando F1-Score $\geq$85\%, Recall $\geq$90\% y Precision $\geq$80\% en el dataset de TechSport (gestión 2025), comparable a benchmarks reportados en literatura científica.
\end{quote}

\subsection*{5.2. Hipótesis Específicas}
\addcontentsline{toc}{subsection}{5.2. Hipótesis Específicas}

\textbf{HE1 -- Fundamentación Teórica:}
\begin{quote}
Al menos el 70\% de los estudios científicos revisados del periodo 2020-2025 reportan que Random Forest alcanza F1-Score $\geq$80\% en detección de fraude financiero, lo que constituye evidencia empírica suficiente para justificar su aplicación en el contexto de TechSport.
\end{quote}

\textbf{HE2 -- Diagnóstico:}
\begin{quote}
El análisis exploratorio del dataset de TechSport revela al menos 3 patrones de fraude recurrentes: tarjetas robadas/clonadas, transacciones duplicadas sospechosas, y comportamientos anómalos de usuarios.
\end{quote}

\textbf{HE3 -- Desarrollo:}
\begin{quote}
Un modelo de Random Forest, entrenado con dataset balanceado y al menos 15 features comportamentales (transaccionales, temporales y de usuario), clasifica transacciones fraudulentas en el validation set temporal (Jul-Ago 2025) con Recall $\geq$90\%, Precision $\geq$80\% y AUC-ROC $\geq$0,90.
\end{quote}

\textbf{HE4 -- Evaluación:}
\begin{quote}
El modelo alcanza en el test set temporal independiente (Sep-Dic 2025, n=5.171.599 transacciones): F1-Score 85-90\%, Recall $\geq$90\%, Precision $\geq$80\%, AUC-ROC $\geq$0,92, tiempo de inferencia $<$200ms. Los intervalos de confianza del 95\% calculados mediante bootstrap confirman la robustez estadística de las métricas.
\end{quote}

\subsection*{5.3. Identificación de las Variables}
\addcontentsline{toc}{subsection}{5.3. Identificación de las Variables}

\textbf{Variable independiente:}

Modelo de Machine Learning (Random Forest).

\textbf{Variable dependiente:}

Fraude transaccional en pagos digitales de TechSport Inc.

\vspace{0.5cm}
A continuación se presenta la operacionalización detallada de cada variable según \textcite[p. 138]{Hernandez2018}.

\vspace{0.5cm}
\textbf{VARIABLE DEPENDIENTE (VD)}
\vspace{0.3cm}

\textbf{Nombre:} Fraude transaccional

\textbf{Definición conceptual:} Actividad ilícita que ocurre cuando una transacción de pago digital es realizada de manera engañosa, sin autorización legítima del titular de la cuenta o método de pago, con el propósito de obtener un beneficio económico indebido.

\textbf{Operacionalización para el estudio:} El fraude transaccional se mide a través de la capacidad del modelo para identificar correctamente transacciones fraudulentas, distinguiéndolas de las legítimas.

\textbf{Definición operacional:} Clasificación binaria de transacciones donde:
\begin{itemize}
    \item \textbf{Fraude (is\_fraud = 1):} Transacción identificada como fraudulenta mediante chargebacks confirmados, disputas resueltas como fraude, o reportes de usuarios verificados
    \item \textbf{No Fraude (is\_fraud = 0):} Transacción legítima sin incidentes reportados
\end{itemize}

\textbf{Dimensiones e indicadores:}

\begin{table}[H]
\centering
\caption{Operacionalización de la Variable Dependiente}
\label{tab:vd_operacionalizacion}
\begin{tabular}{|p{2.5cm}|p{3.5cm}|p{4cm}|p{2cm}|}
\hline
\rowcolor{headerblue}
\textcolor{white}{\textbf{Dimensión}} & \textcolor{white}{\textbf{Indicador}} & \textcolor{white}{\textbf{Fórmula/Medición}} & \textcolor{white}{\textbf{Meta}} \\
\hline
Sensibilidad & Recall (TVP) & TP / (TP + FN) & $\geq$ 90\% \\
\hline
Exactitud & Precision (VPP) & TP / (TP + FP) & $\geq$ 80\% \\
\hline
Balance & F1-Score & 2$\times$(Prec$\times$Rec)/(Prec+Rec) & $\geq$ 85\% \\
\hline
Discriminación & AUC-ROC & Área bajo curva ROC & $\geq$ 0,92 \\
\hline
Errores & Tasa Falsos Positivos & FP / (FP + TN) & $<$ 5\% \\
\hline
Eficiencia & Tiempo inferencia & Milisegundos/transacción & $<$ 200ms \\
\hline
\end{tabular}
\end{table}

\textbf{Escala de medición:}
\begin{itemize}
    \item Tipo: Nominal dicotómica (Fraude/No Fraude)
    \item Métricas: Razón (porcentajes 0-100\%)
\end{itemize}

\vspace{0.5cm}
\textbf{VARIABLE INDEPENDIENTE (VI)}
\vspace{0.3cm}

\textbf{Nombre:} Modelo de Machine Learning (Random Forest)

\textbf{Definición conceptual:} Algoritmo de aprendizaje automático supervisado de tipo ensemble que combina múltiples árboles de decisión entrenados con subconjuntos aleatorios de datos y características, generando predicciones por votación mayoritaria.

\textbf{Definición operacional:} Modelo de clasificación binaria implementado con la biblioteca scikit-learn de Python, que produce:
\begin{enumerate}
    \item Probabilidad de fraude (score entre 0 y 1)
    \item Clasificación final (0 o 1) basada en umbral optimizado
\end{enumerate}

\textbf{Dimensiones e indicadores:}

\begin{table}[H]
\centering
\caption{Operacionalización de la Variable Independiente}
\label{tab:vi_operacionalizacion}
\begin{tabular}{|p{3cm}|p{4cm}|p{4.5cm}|}
\hline
\rowcolor{headerblue}
\textcolor{white}{\textbf{Dimensión}} & \textcolor{white}{\textbf{Indicador}} & \textcolor{white}{\textbf{Valores/Rango}} \\
\hline
Arquitectura & Algoritmo base & Random Forest (ensemble) \\
\hline
Complejidad & n\_estimators & 100 - 500 árboles \\
\hline
Profundidad & max\_depth & 10 - 20 niveles \\
\hline
Regularización & min\_samples\_split & 2 - 10 muestras \\
\hline
Balanceo & class\_weight & 'balanced' o SMOTE \\
\hline
Características & Número de features & $\geq$ 15 variables \\
\hline
Eficiencia & Tiempo de inferencia & $<$ 200 ms/transacción \\
\hline
\end{tabular}
\end{table}

\vspace{0.5cm}
\textbf{VARIABLES INTERVINIENTES (CONTROL)}
\vspace{0.3cm}

Variables que podrían afectar la relación VI $\rightarrow$ VD y deben controlarse:

\begin{table}[H]
\centering
\caption{Variables Intervinientes}
\label{tab:variables_intervinientes}
\begin{tabular}{|p{4cm}|p{2cm}|p{6cm}|}
\hline
\rowcolor{headerblue}
\textcolor{white}{\textbf{Variable}} & \textcolor{white}{\textbf{Tipo}} & \textcolor{white}{\textbf{Categorías/Valores}} \\
\hline
Canal de pago & Nominal & Web, App Móvil, POS, Transferencia, Terminal \\
\hline
Gateway de pago & Nominal & Stripe, CardConnect, Kushki, AzulPay, RazorPay, BAC, Otros \\
\hline
Tipo de transacción & Nominal & Reserva, Membresía, Clínica, Cargo recurrente \\
\hline
País/Región & Nominal & USA, Latam, Europa, Otros \\
\hline
Moneda & Nominal & USD, EUR, MXN, COP, otros \\
\hline
\end{tabular}
\end{table}

% ==================================================================================
% 6. DISEÑO METODOLÓGICO
% ==================================================================================

\section*{6. Diseño Metodológico}
\addcontentsline{toc}{section}{6. Diseño Metodológico}

\subsection*{6.1. Tipo, enfoque y alcance de la investigación}
\addcontentsline{toc}{subsection}{6.1. Tipo, enfoque y alcance de la investigación}

\subsubsection*{6.1.1. Tipo de investigación}

\textbf{a) Aplicada}

Según \textcite[p. 29]{Hernandez2018}: \textit{``La investigación aplicada tiene como propósito resolver problemas prácticos''}. El presente trabajo de investigación aplica el tipo de investigación aplicada, formulando una solución concreta al problema de fraude transaccional en TechSport. La propuesta genera un modelo evaluable cuyos resultados tienen utilidad práctica y pueden transferirse a organizaciones similares.

\subsubsection*{6.1.2. Enfoque de la investigación}

La investigación tiene un \textbf{enfoque cuantitativo}. Según \textcite[p. 4]{Hernandez2018}: \textit{``El enfoque cuantitativo utiliza la recolección de datos para probar hipótesis con base en la medición numérica y el análisis estadístico''}.

En esta investigación se analizan datos numéricos (15,6M+ transacciones), se utilizan métricas cuantitativas (Precision, Recall, F1-Score, AUC-ROC), se aplican técnicas estadísticas (intervalos de confianza, bootstrap), se prueban hipótesis con umbrales específicos (F1 $\geq$ 85\%), y los resultados son replicables y verificables.

\subsubsection*{6.1.3. Alcance de la investigación}

El alcance del presente trabajo de investigación es \textbf{correlacional-explicativo}. Presenta un componente correlacional al establecer la relación entre la variable independiente (Modelo Random Forest) y la variable dependiente (Fraude transaccional). Asimismo, presenta un componente explicativo al plantear una hipótesis de relación causal: la aplicación del modelo Random Forest permite detectar fraude con F1-Score $\geq$ 85\%.

\subsubsection*{6.1.4. Diseño de investigación}

El diseño de investigación es \textbf{no experimental, transversal y retrospectivo}. Según \textcite[p. 152]{Hernandez2018}: \textit{``En un estudio no experimental no se genera ninguna situación, sino que se observan situaciones ya existentes''}.

Es no experimental porque las transacciones ya ocurrieron y no se manipulan variables en tiempo real. Es transversal porque los datos se extraen una sola vez (snapshot de gestión 2025); la división Train/Validation/Test es una estrategia de validación de Machine Learning, no un diseño longitudinal. Es retrospectivo porque los datos corresponden a transacciones ya ocurridas y las etiquetas de fraude fueron asignadas después de los eventos mediante chargebacks confirmados.

\subsection*{6.2. Delimitación de la Investigación}
\addcontentsline{toc}{subsection}{6.2. Delimitación de la Investigación}

\textbf{Delimitación temática:} La investigación se limita al estudio de la detección de fraude en pagos digitales mediante Machine Learning supervisado, específicamente utilizando el algoritmo Random Forest (ensemble learning). Los tipos de fraude incluidos son: tarjetas robadas o clonadas, transacciones duplicadas sospechosas, y comportamientos anómalos de usuarios. La investigación no contempla el tratamiento de lavado de dinero, detección en tiempo real (streaming), modelos de Deep Learning, ni análisis de imágenes o documentos de identidad.

\textbf{Delimitación espacial:} La investigación se efectúa en la empresa TechSport Inc., con sede principal en Miami, Florida, Estados Unidos, y operación internacional en múltiples países de América y Europa. La evaluación se realiza sobre los datos de transacciones procesadas a través de sus pasarelas de pago integradas.

\textbf{Delimitación temporal:} La investigación se realizará durante el lapso de tres meses. El período de datos analizado corresponde a la gestión 2025 (enero a diciembre). La propuesta debe ser ajustada cuando las condiciones del mercado de pagos digitales o las técnicas de fraude sufran modificaciones significativas.

\subsection*{6.3. Población y Muestra}
\addcontentsline{toc}{subsection}{6.3. Población y Muestra}

Para la selección de la muestra se emplea un método de censo completo, justificado bajo la necesidad de analizar la totalidad de transacciones debido al desbalance de clases inherente a los problemas de detección de fraude (donde las transacciones fraudulentas representan menos del 1\% del total).

La población de estudio comprende la totalidad de transacciones de pago procesadas por TechSport durante la gestión 2025, correspondiente a \textbf{15.671.512 registros}. La muestra considera los criterios de disponibilidad técnica (datos almacenados en base de datos ClickHouse), capacidad computacional para procesar el volumen completo, y etiquetas de fraude validadas por el equipo de contabilidad mediante chargebacks confirmados.

\textbf{Partición temporal del dataset:}

\begin{table}[H]
\centering
\caption{División temporal del dataset}
\label{tab:particion_temporal}
\begin{tabular}{|p{3.5cm}|p{2.5cm}|p{3cm}|p{4cm}|}
\hline
\rowcolor{headerblue}
\textcolor{white}{\textbf{Conjunto}} & \textcolor{white}{\textbf{Período}} & \textcolor{white}{\textbf{Porcentaje}} & \textcolor{white}{\textbf{Transacciones}} \\
\hline
Training set & Ene-Jun 2025 & 50\% & 7.835.756 \\
\hline
Validation set & Jul-Ago 2025 & 17\% & 2.664.157 \\
\hline
Test set & Sep-Dic 2025 & 33\% & 5.171.599 \\
\hline
\rowcolor{lightgreen}
\textbf{Total} & \textbf{Ene-Dic 2025} & \textbf{100\%} & \textbf{15.671.512} \\
\hline
\end{tabular}
\end{table}

\subsection*{6.4. Métodos y Técnicas de Investigación}
\addcontentsline{toc}{subsection}{6.4. Métodos y Técnicas de Investigación}

\textbf{Métodos de investigación:}

\begin{itemize}
    \item \textbf{Método analítico-sintético:} Se descompone el problema de detección de fraude en componentes manejables (preprocesamiento, feature engineering, entrenamiento, evaluación), analizando cada etapa individualmente para luego integrarlas en un pipeline coherente.

    \item \textbf{Método inductivo-deductivo:} A partir de la observación de patrones específicos en transacciones fraudulentas históricas (inducción), se formulan hipótesis generales sobre características predictivas de fraude, las cuales se validan mediante experimentación (deducción).

    \item \textbf{Método estadístico:} Se emplean técnicas estadísticas para análisis exploratorio de datos, validación de hiperparámetros y cálculo de intervalos de confianza mediante bootstrap.
\end{itemize}

\textbf{Técnicas de recolección de datos:}

\begin{itemize}
    \item \textbf{Extracción de datos históricos:} Consultas SQL a base de datos ClickHouse
    \item \textbf{Análisis documental:} Revisión de documentación técnica interna de TechSport
    \item \textbf{Revisión de literatura científica:} Búsqueda en bases académicas (IEEE, ACM, Scopus)
\end{itemize}

\textbf{Instrumentos de investigación:}

\begin{itemize}
    \item Scripts de extracción de datos (Python/SQL)
    \item Pipeline de preprocesamiento (pandas, numpy, scikit-learn)
    \item Framework de modelado (scikit-learn: RandomForestClassifier)
    \item Herramientas de análisis exploratorio (matplotlib, seaborn)
\end{itemize}

\subsection*{6.5. Validez y Confiabilidad}
\addcontentsline{toc}{subsection}{6.5. Validez y Confiabilidad}

\textbf{Validez de contenido:} Las features del modelo fueron seleccionadas mediante revisión de literatura científica (OE1), asegurando que representan dimensiones validadas empíricamente para detección de fraude.

\textbf{Validez de criterio:} La variable target (\texttt{is\_fraud}) fue etiquetada mediante chargebacks confirmados por bancos emisores, disputas resueltas a favor del usuario, y reportes verificados por equipo de contabilidad.

\textbf{Validez de constructo:} La capacidad discriminativa se evalúa mediante AUC-ROC, métrica estándar que mide la habilidad del modelo para distinguir entre clases.

\textbf{Confiabilidad:} La estabilidad temporal se garantiza evaluando el modelo en tres períodos temporales independientes (Train, Validation, Test). Los intervalos de confianza al 95\% se calculan mediante bootstrap con 1000 iteraciones, asegurando la robustez estadística de las métricas reportadas.

\subsection*{6.6. Análisis de los Datos}
\addcontentsline{toc}{subsection}{6.6. Análisis de los Datos}

La información obtenida mediante la extracción de datos históricos permite obtener un panorama de la situación actual del objeto de investigación. El análisis de los datos se realiza en las siguientes etapas:

\begin{itemize}
    \item \textbf{Análisis exploratorio de datos (EDA):} Examen de distribuciones univariadas, identificación de correlaciones entre variables, detección de outliers, y caracterización de patrones de fraude.

    \item \textbf{Preprocesamiento y transformación:} Limpieza de datos, normalización de variables numéricas, codificación de variables categóricas, y creación de features derivadas evitando data leakage.

    \item \textbf{Balanceo de clases:} Evaluación de estrategias SMOTE, class\_weight='balanced', o combinación híbrida para manejar el desbalance inherente en problemas de detección de fraude.

    \item \textbf{Entrenamiento y optimización:} Random Forest con optimización de hiperparámetros mediante GridSearchCV o RandomizedSearchCV.

    \item \textbf{Evaluación del desempeño:} Métricas en test set temporal independiente con intervalos de confianza del 95\% mediante bootstrap (1000 muestras).

    \item \textbf{Interpretabilidad:} Análisis de importancia de features mediante \texttt{feature\_importances\_} de Random Forest.
\end{itemize}

\subsection*{6.7. Matriz de Consistencia}
\addcontentsline{toc}{subsection}{6.7. Matriz de Consistencia}

\begin{table}[H]
\centering
\caption{Matriz de Consistencia Metodológica}
\label{tab:matriz_consistencia}
\footnotesize
\begin{tabular}{|p{2.8cm}|p{3.2cm}|p{3.2cm}|p{1.8cm}|p{2.5cm}|}
\hline
\rowcolor{headerblue}
\textcolor{white}{\textbf{Problema}} & \textcolor{white}{\textbf{Objetivo}} & \textcolor{white}{\textbf{Hipótesis}} & \textcolor{white}{\textbf{Variables}} & \textcolor{white}{\textbf{Indicadores}} \\
\hline
\textbf{PG:} ¿Cuál es la capacidad predictiva de RF? & \textbf{OG:} Evaluar capacidad predictiva del modelo RF & \textbf{HG:} Modelo alcanza F1$\geq$85\%, Recall$\geq$90\%, Precision$\geq$80\% & VI: Modelo RF \newline VD: Fraude transaccional & F1, Recall, Precision, AUC-ROC \\
\hline
\textbf{PE1:} ¿Fundamento teórico de RF? & \textbf{OE1:} Fundamentar teóricamente & \textbf{HE1:} $\geq$70\% estudios reportan F1$\geq$80\% & Marco teórico & \% estudios, métricas \\
\hline
\textbf{PE2:} ¿Patrones de fraude en dataset? & \textbf{OE2:} Caracterizar patrones & \textbf{HE2:} $\geq$3 patrones identificados & Diagnóstico & Patrones, distribuciones \\
\hline
\textbf{PE3:} ¿Cómo desarrollar modelo? & \textbf{OE3:} Desarrollar pipeline & \textbf{HE3:} $\geq$15 features, Recall$\geq$90\% & Modelo RF & Features, hiperparámetros \\
\hline
\textbf{PE4:} ¿Desempeño en test set? & \textbf{OE4:} Evaluar métricas con IC95\% & \textbf{HE4:} F1 85-90\%, IC95\% bootstrap & Métricas & IC 95\%, benchmarks \\
\hline
\end{tabular}
\end{table}

\cleardoublepage
