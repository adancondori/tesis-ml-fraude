% ==================================================================================
% PERFIL DE TESIS - CORREGIDO SEGÚN DRA. ROSARIO MARTÍNEZ Y SAMPIERI
% Maestría en Dirección Estratégica en Ingeniería de Software
% UAGRM - Gestión 2025
% ==================================================================================

% ==================================================================================
% SECCIONES DEL PERFIL DE INVESTIGACIÓN (1-5)
% Según plantilla UAGRM y Método AQP/CCA
% ==================================================================================

% ==================================================================================
% 1. ANTECEDENTES DEL PROBLEMA
% ==================================================================================

\section*{1. Antecedentes del Problema}
\addcontentsline{toc}{section}{1. Antecedentes del Problema}

En la economía digital global, el crecimiento sostenido de los pagos electrónicos ha traído consigo un desafío crítico: el aumento exponencial de fraudes financieros sofisticados. A medida que las transacciones digitales migran hacia plataformas móviles y web, también lo hacen las técnicas utilizadas por actores maliciosos, quienes desarrollan métodos cada vez más complejos para evadir los sistemas de seguridad tradicionales. Este fenómeno se ha visto acelerado por el auge de servicios fintech y soluciones SaaS (Software as a Service), que requieren arquitecturas distribuidas capaces de procesar millones de transacciones diarias de forma segura y eficiente.

Según \textcite{HernandezAros2024}, los sistemas de detección de fraude basados en reglas estáticas y revisión posterior han quedado obsoletos, dado que los ataques actuales son dinámicos, adaptativos y evolucionan más rápidamente que la capacidad de actualización manual de reglas. Por ello, diversos estudios proponen el uso de técnicas de inteligencia artificial (IA) y aprendizaje automático para analizar patrones transaccionales en tiempo real y detectar comportamientos anómalos con mayor eficacia y precisión.

\textcite{Hafez2025} demuestran, mediante una revisión sistemática de la literatura, que los modelos de aprendizaje automático superan significativamente en precisión a los enfoques tradicionales en la detección de fraude con tarjetas de crédito, reportando F1-Scores entre 85\% y 94\% en contextos reales. Estos modelos destacan por su capacidad de adaptación continua y eficiencia en el procesamiento de grandes volúmenes de datos. Sin embargo, su implementación efectiva requiere arquitecturas técnicas robustas, capaces de operar bajo estrictos estándares de seguridad como PCI DSS (Payment Card Industry Data Security Standard) o el Marco de Ciberseguridad del NIST \parencite{NIST2024}.

En el continente americano, tanto América Latina como Estados Unidos enfrentan retos importantes en materia de seguridad transaccional digital. En América Latina, la rápida adopción de tecnologías digitales ha incrementado significativamente la exposición a fraudes financieros, sin que ello haya estado acompañado por un desarrollo equivalente en mecanismos de prevención y detección. \textcite{OEABID2020} documentan brechas críticas en capacidades de monitoreo, análisis de amenazas y respuesta operativa en la región. Asimismo, la fragmentación del ecosistema —derivada de la diversidad de medios de pago, regulaciones dispares entre países y niveles disímiles de madurez tecnológica— crea un entorno propicio para la aparición de fraudes que evolucionan más rápido que los controles existentes.

En contraste, aunque Estados Unidos dispone de marcos regulatorios avanzados y tecnologías más maduras para la detección de fraudes, también enfrenta limitaciones importantes. El volumen masivo de transacciones procesadas diariamente, la creciente sofisticación de los ataques cibernéticos y la dependencia persistente de sistemas basados en reglas estáticas limitan la capacidad de respuesta efectiva frente a amenazas emergentes. Casos documentados en empresas tecnológicas estadounidenses ponen de manifiesto que, incluso en contextos con alta madurez digital, persisten vulnerabilidades relevantes en los sistemas actuales de detección de fraude.

En este contexto se ubica la empresa TechSport, una plataforma SaaS con presencia internacional, especializada en la gestión integral de instalaciones deportivas de raqueta (tenis, pádel, pickleball, basketball). La compañía ha integrado más de diez pasarelas de pago diferentes (entre ellas Stripe, CardConnect, Kushki, AzulPay, RazorPay y BAC), lo que le permite operar con múltiples monedas y a través de diversos canales (web, aplicación móvil y puntos de venta físicos). No obstante, esta diversidad tecnológica ha generado una arquitectura fragmentada, carente de un sistema centralizado e inteligente de detección de fraude. Actualmente, TechSport no dispone de mecanismos basados en aprendizaje automático para identificar anomalías transaccionales de forma proactiva, ni de modelos predictivos capaces de alertar sobre patrones sospechosos antes de que se consumen las transacciones fraudulentas.

Esta situación representa un riesgo operacional significativo para la empresa, tanto por las potenciales pérdidas económicas directas (fraudes consumados, chargebacks, disputas) como por el impacto negativo en la experiencia del usuario (rechazos incorrectos de pagos legítimos) y el posible incumplimiento de normativas internacionales de seguridad y protección de datos. Adicionalmente, la ausencia de un sistema inteligente de detección genera una carga operativa elevada en los equipos de soporte y contabilidad, quienes deben gestionar manualmente reclamos, disputas y análisis post-mortem de transacciones fraudulentas.

Un diagnóstico técnico interno ha identificado como causas fundamentales del problema: (i) la ausencia de una arquitectura unificada para la gestión del riesgo transaccional, que permita correlacionar comportamientos entre diferentes gateways y canales; (ii) la falta de automatización en los procesos de evaluación de fraude, dependiendo exclusivamente de reglas estáticas que requieren actualización manual constante; y (iii) la carencia de una gobernanza efectiva sobre las integraciones entre sistemas y APIs, lo que dificulta la trazabilidad y el análisis contextual de las transacciones. Estas deficiencias técnicas aumentan la probabilidad de errores operativos, dificultan la escalabilidad del sistema y reducen significativamente la capacidad de adaptación ante nuevas modalidades de fraude.

Las consecuencias de esta situación problemática incluyen un incremento progresivo en la tasa de falsos positivos (rechazos incorrectos de pagos legítimos), lo que deteriora la experiencia del usuario y genera pérdida de confianza en la plataforma; una detección post-mortem de fraudes (identificados semanas o meses después mediante chargebacks), que imposibilita la prevención efectiva; y una disminución en la competitividad de la empresa frente a plataformas fintech que sí implementan soluciones basadas en inteligencia artificial.

Hasta donde se ha podido verificar mediante revisión documental y análisis institucional, no existen proyectos anteriores ni en ejecución en la empresa TechSport que propongan una solución basada en técnicas de aprendizaje automático para la detección de fraude en pagos transaccionales. En este contexto, se considera necesario y viable implementar una solución técnica que permita optimizar el análisis de transacciones mediante modelos supervisados de Machine Learning, ajustados a las condiciones reales de operación de la empresa y validados con datos históricos de producción.

% ==================================================================================
% 2. FORMULACIÓN DEL PROBLEMA
% ==================================================================================

\section*{2. Formulación del Problema}
\addcontentsline{toc}{section}{2. Formulación del Problema}

La arquitectura tecnológica de pagos multicanal implementada actualmente en la empresa TechSport presenta limitaciones estructurales y técnicas que dificultan la detección oportuna de transacciones fraudulentas y anómalas. Esta situación incrementa los riesgos operacionales y compromete tanto la seguridad de las transacciones como la experiencia del usuario.

\textbf{¿Cómo mejorar la detección de transacciones fraudulentas y anómalas en pagos digitales de la empresa TechSport durante la gestión 2025?}

\subsection*{2.1. Objeto de Estudio}
\addcontentsline{toc}{subsection}{2.1. Objeto de Estudio}

Transacciones fraudulentas y anómalas en pagos digitales procesados por plataformas SaaS multicanal.

\subsection*{2.2. Campo de Acción}
\addcontentsline{toc}{subsection}{2.2. Campo de Acción}

Implementación de modelos de Machine Learning supervisados para la detección de fraude en pagos transaccionales en la empresa TechSport durante la gestión 2025.

% ==================================================================================
% 3. OBJETIVOS DE LA INVESTIGACIÓN
% ==================================================================================

\section*{3. Objetivos de la Investigación}
\addcontentsline{toc}{section}{3. Objetivos de la Investigación}

\subsection*{3.1. Objetivo General}
\addcontentsline{toc}{subsection}{3.1. Objetivo General}

Implementar un modelo de Machine Learning supervisado basado en Random Forest para la detección de transacciones fraudulentas y anómalas en pagos digitales, mediante el análisis de datos históricos (15.4M+ transacciones de gestión 2025), feature engineering evitando data leakage, balanceo de clases adaptativo y validación estratificada (Train 70\%, Validation 15\%, Test 15\%), logrando un F1-Score $\geq$ 85\%, Recall $\geq$ 90\% y Precision $\geq$ 80\%, demostrando desempeño comparable o superior a benchmarks reportados en literatura científica, en la empresa TechSport, gestión 2025.

\subsection*{3.2. Objetivos Específicos}
\addcontentsline{toc}{subsection}{3.2. Objetivos Específicos}

\begin{enumerate}
    \item Fundamentar teóricamente los modelos de Machine Learning supervisados aplicados a detección de fraude en pagos digitales, con énfasis en Random Forest y enfoques de ensemble learning, revisando la literatura científica del periodo 2020-2025, así como las métricas de evaluación de desempeño (Precision, Recall, F1-Score, AUC-ROC), técnicas de feature engineering y estrategias de balanceo de clases, para sustentar la base conceptual y técnica de la investigación.

    \item Diagnosticar la situación actual del sistema de detección de fraude de TechSport mediante análisis exploratorio del dataset histórico de gestión 2025, documentando el proceso de etiquetado de transacciones fraudulentas realizado por el equipo de contabilidad de la empresa y caracterizando los tres principales patrones de fraude presentes: (i) tarjetas robadas o clonadas, (ii) transacciones duplicadas sospechosas, y (iii) comportamientos anómalos de usuarios.

    \item Desarrollar e implementar un modelo de Machine Learning supervisado basado en Random Forest para la detección de transacciones fraudulentas y anómalas, mediante un pipeline que incluya: (i) preprocesamiento de 15.4M+ transacciones con manejo de valores faltantes y outliers, (ii) feature engineering de al menos 15 features comportamentales evitando data leakage, (iii) estrategia de balanceo de clases adaptativo (SMOTE o class weights según distribución), (iv) división estratificada del dataset (Train 70\%, Validation 15\%, Test 15\%), y (v) optimización de hiperparámetros mediante Grid Search o Random Search.

    \item Evaluar el desempeño del modelo de Machine Learning seleccionado mediante métricas de clasificación (Precision, Recall, F1-Score, AUC-ROC, tasa de falsos positivos, tiempo de inferencia) aplicadas sobre el test set temporal independiente (transacciones de 2025 = 15.5M transacciones), documentando el desempeño absoluto del modelo y comparándolo con benchmarks de la literatura científica, calculando intervalos de confianza del 95\% mediante bootstrap (1000 muestras) para validar la robustez de las métricas obtenidas.
\end{enumerate}

% ==================================================================================
% 4. JUSTIFICACIÓN
% ==================================================================================

\section*{4. Justificación de la Investigación}
\addcontentsline{toc}{section}{4. Justificación de la Investigación}

\textbf{Justificación teórica.} La presente investigación se enmarca en los fundamentos del aprendizaje automático supervisado y su aplicación en entornos transaccionales digitales. La literatura científica ha demostrado de forma consistente que los modelos de Machine Learning ofrecen ventajas significativas frente a los enfoques tradicionales de detección de fraude, principalmente por su capacidad para identificar patrones complejos no lineales, adaptarse continuamente a nuevos comportamientos fraudulentos y reducir sustancialmente la tasa de errores en la clasificación de eventos anómalos \parencite{Hafez2025}. Esta investigación pretende contribuir al cuerpo de conocimientos en el campo de la inteligencia artificial aplicada a la seguridad digital y protección financiera, generando evidencia empírica sobre la efectividad de modelos supervisados en escenarios reales de pagos electrónicos multicanal. Al aplicar estos enfoques teóricos en una empresa tipo SaaS del sector deportivo, se amplía la aplicabilidad de estos modelos más allá del ámbito académico, promoviendo su validación práctica en contextos empresariales concretos y contribuyendo al desarrollo de buenas prácticas en ingeniería de software aplicada a seguridad financiera.

\textbf{Justificación práctica.} El estudio responde a una necesidad operativa concreta y urgente de la empresa TechSport, que enfrenta dificultades significativas para identificar transacciones fraudulentas con los sistemas actuales basados en reglas estáticas. La ausencia de herramientas inteligentes de análisis y clasificación automática de comportamiento transaccional limita severamente la capacidad de respuesta preventiva ante fraudes, eleva considerablemente el número de falsos positivos (rechazos incorrectos de pagos legítimos) y genera una carga operativa excesiva en los equipos de soporte y contabilidad. La propuesta de implementar un modelo de aprendizaje automático supervisado busca mejorar sustancialmente la precisión en la detección de anomalías, reducir errores operativos, minimizar pérdidas económicas y fortalecer los mecanismos internos de control y gobernanza, con un enfoque realista, contextualizado y técnicamente viable en un plazo de dos meses. Además, el modelo desarrollado podría adaptarse y replicarse en otras plataformas tecnológicas con arquitecturas similares (SaaS multicanal deportivas o fintech), lo cual le otorga un valor de transferencia tecnológica relevante para el sector.

\textbf{Justificación económica.} Esta investigación se justifica económicamente en la medida en que los sistemas inteligentes de detección de fraude no solo buscan mitigar pérdidas directas por actividades ilícitas consumadas, sino también prevenir costos indirectos asociados a la gestión reactiva de incidentes, sanciones regulatorias por incumplimiento de normativas, deterioro reputacional y pérdida progresiva de confianza de los usuarios institucionales (clubes deportivos). La empresa TechSport, al operar en múltiples mercados geográficos y manejar volúmenes transaccionales elevados (15M+ transacciones en gestión 2025), se encuentra expuesta a riesgos financieros que pueden traducirse en impactos económicos significativos y sostenidos en el tiempo. Implementar un sistema predictivo basado en datos históricos y validado científicamente contribuirá a optimizar la asignación de recursos, reducir costos operativos de gestión manual, proteger la estabilidad financiera de la organización y generar un retorno de inversión (ROI) positivo a través de decisiones más informadas, automatizadas y auditables.

\textbf{Justificación metodológica.} El estudio adopta un enfoque cuantitativo con diseño cuasiexperimental retrospectivo, de tipo aplicado y alcance descriptivo-correlacional-comparativo. Se desarrollará un modelo de aprendizaje automático supervisado (Random Forest) entrenado con datos históricos reales de la empresa (dataset de 15.4M+ transacciones de gestión 2025), y se evaluará su desempeño mediante métricas técnicas estandarizadas como Precision, Recall, F1-Score y AUC-ROC. El uso exclusivo de gestión 2025 garantiza homogeneidad temporal y evita data drift entre períodos, permitiendo validación estratificada robusta (70/15/15). Esta aproximación metodológica, fundamentada en los principios de Sampieri para investigación cuantitativa, permitirá validar la viabilidad del modelo en un entorno controlado (test set independiente) y bajo condiciones reales del negocio, sin necesidad de modificar inicialmente el sistema productivo ni generar riesgos operacionales. Asimismo, se garantiza la reproducibilidad de los resultados mediante la documentación exhaustiva del pipeline de preprocesamiento y feature engineering, y la coherencia con estándares técnicos y académicos reconocidos internacionalmente, asegurando que las conclusiones derivadas del estudio estén sustentadas en evidencia empírica sólida, verificable y auditable.

\textbf{Justificación de viabilidad temporal.} La investigación ha sido diseñada específicamente para ser ejecutada en un plazo de dos meses (12 semanas), con un cronograma realista que contempla 30-40 horas de dedicación semanal. El alcance del estudio ha sido deliberadamente acotado para excluir componentes que excederían este plazo (implementación en producción en tiempo real, arquitecturas de streaming, deep learning, análisis de tipos complejos de fraude como lavado de dinero), enfocándose exclusivamente en el desarrollo, validación y evaluación del modelo de Machine Learning en ambiente controlado. El dataset ya se encuentra disponible y etiquetado por la empresa, la infraestructura computacional (AWS) está configurada, y los objetivos específicos están alineados con hitos semanales verificables, lo que garantiza la factibilidad de completar exitosamente la investigación en el tiempo establecido.

% ==================================================================================
% 5. FORMULACIÓN DE LA CONSTRUCCIÓN TEÓRICA. HIPÓTESIS PARA DEFENDER
% ==================================================================================

\section*{5. Formulación de la Construcción Teórica. Hipótesis para Defender}
\addcontentsline{toc}{section}{5. Formulación de la Construcción Teórica. Hipótesis para Defender}

La implementación de un modelo de Machine Learning supervisado basado en Random Forest alcanza un F1-Score mínimo del 85\%, con Recall $\geq$ 90\% y Precision $\geq$ 80\%, en la detección de transacciones fraudulentas y anómalas del test set temporal (transacciones de 2025 = 15.5M transacciones) de TechSport, demostrando desempeño comparable o superior a benchmarks reportados en literatura científica (F1-Scores de 85-94\% según \textcite{Hafez2025}) y manteniendo un tiempo de inferencia inferior a 200 milisegundos por transacción.

\subsection*{5.1. Identificación de las Variables}
\addcontentsline{toc}{subsection}{5.1. Identificación de las Variables}

\textbf{Variable Independiente (VI):} Modelo de Machine Learning implementado.

\textbf{Definición conceptual:} Algoritmo computacional basado en aprendizaje automático supervisado, capaz de analizar datos históricos de transacciones etiquetadas para identificar patrones asociados a fraude y predecir la probabilidad de que nuevas transacciones sean fraudulentas o legítimas.

\textbf{Definición operacional:} Modelo de clasificación binaria (Fraude/No Fraude) basado en Random Forest, entrenado con dataset histórico de TechSport (transacciones de 2024), que genera un score de riesgo para cada transacción y una clasificación final basada en un umbral optimizado.

\textbf{Indicadores:}
\begin{itemize}
    \item Algoritmo seleccionado: Random Forest
    \item Hiperparámetros optimizados: max\_depth (10-20), n\_estimators (100-500), class\_weight (balanceado)
    \item Tiempo de entrenamiento (minutos)
    \item Tiempo de inferencia por transacción (milisegundos)
    \item Tamaño del modelo serializado (MB)
\end{itemize}

\vspace{0.5cm}

\textbf{Variable Dependiente (VD):} Transacciones fraudulentas y anómalas en pagos digitales.

\textbf{Definición conceptual:} Conjunto de transacciones de pago procesadas por TechSport que presentan comportamientos sospechosos, patrones atípicos o características asociadas a actividad fraudulenta, que pueden resultar en pérdidas económicas, chargebacks o afectación de la seguridad financiera de la plataforma.

\textbf{Definición operacional:} Transacciones clasificadas como fraudulentas o anómalas según el proceso de etiquetado realizado por el equipo de contabilidad de TechSport, basado en: (i) chargebacks confirmados por instituciones financieras, (ii) disputas resueltas como fraude, (iii) reportes de usuarios afectados verificados, y (iv) revisión manual de transacciones sospechosas. El tiempo de etiquetado varía entre 0 días (detección inmediata) hasta 5 meses después de la transacción (chargebacks tardíos).

\textbf{Indicadores:}
\begin{itemize}
    \item Tasa de fraude detectado (\%)
    \item Tasa de fraude NO detectado (\%)
    \item Pérdidas económicas por fraude (USD)
    \item Precisión de clasificación (Precision, \%)
    \item Sensibilidad de detección (Recall, \%)
    \item F1-Score (balance precision-recall)
    \item Tasa de falsos positivos (\%)
    \item AUC-ROC (área bajo curva ROC)
    \item Tiempo promedio de detección (segundos)
\end{itemize}

\vspace{0.5cm}

\textbf{Variables Intervinientes:} Canal de pago (Web, App Móvil, POS), Tipo de transacción (Reserva, Membresía, Clínica, Cargo recurrente, One-time), Gateway de pago (Stripe, CardConnect, Kushki, AzulPay, RazorPay, BAC, otros).

\subsection*{5.2. Hipótesis Específicas}
\addcontentsline{toc}{subsection}{5.2. Hipótesis Específicas}

\textbf{HE1 (Fundamentación Teórica):} La revisión de literatura científica del periodo 2020-2025 valida que los modelos de Machine Learning supervisados, particularmente los enfoques de ensemble learning como Random Forest, constituyen un marco teórico-técnico respaldado por al menos 20 estudios científicos para la detección de fraude en pagos digitales, reportando F1-Scores entre 85-94\% y superando las limitaciones de sistemas basados en reglas estáticas en términos de adaptabilidad (capacidad de aprender nuevos patrones), precisión (menor tasa de falsos positivos/negativos) y escalabilidad (procesamiento de grandes volúmenes).

\textbf{HE2 (Diagnóstico):} Se espera que el sistema actual de TechSport basado en reglas estáticas presente limitaciones operativas al analizar el dataset histórico de gestión 2025, evidenciadas por: (i) presencia de transacciones fraudulentas no detectadas oportunamente (identificadas post-mortem mediante chargebacks con retraso de 0-5 meses), (ii) necesidad de actualización manual constante de reglas sin capacidad de aprendizaje automático, y (iii) ausencia de correlación cruzada entre comportamientos en diferentes gateways y canales. El análisis exploratorio del dataset revelará al menos 3 patrones de fraude recurrentes que el sistema actual no detecta eficazmente.

\textbf{HE3 (Desarrollo):} Un modelo de Machine Learning supervisado basado en Random Forest, entrenado con dataset balanceado mediante SMOTE o class weights (según distribución de clases) y al menos 15 features de comportamiento transaccional (monto normalizado, frecuencia de transacciones, geolocalización IP, canal, gateway, velocidad transaccional, tiempo desde última transacción, hora del día, día de la semana, historial de chargebacks, ratio monto/promedio histórico), puede clasificar transacciones fraudulentas en el test set temporal (2025) con un Recall mínimo del 90\%, Precision mínima del 80\% y AUC-ROC $\geq$ 0.92, evitando data leakage mediante el uso exclusivo de información disponible al momento de la transacción.

\textbf{HE4 (Evaluación):} El modelo de Machine Learning implementado alcanza un F1-Score de 85-90\% en el test set temporal independiente (transacciones de 2025 = 15.5M transacciones), con Recall $\geq$ 90\%, Precision $\geq$ 80\% y AUC-ROC $\geq$ 0.92, demostrando desempeño comparable o superior a benchmarks reportados en literatura científica (\textcite{Hafez2025} reporta F1-Scores de 85-94\% en contextos similares de detección de fraude con tarjetas de crédito). El modelo mantiene un tiempo de inferencia promedio inferior a 200 milisegundos por transacción, demostrando viabilidad técnica para potencial implementación en producción. Los intervalos de confianza del 95\% (calculados mediante bootstrap con 1000 muestras) confirman la robustez y estabilidad estadística de las métricas obtenidas.

\cleardoublepage
