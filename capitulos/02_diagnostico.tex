% ==================================================================================
% CAPÍTULO 2: DIAGNÓSTICO Y ANÁLISIS DE RESULTADOS
% Triangulación: PE2 → OE2 → HE2
% ==================================================================================

\chapter{Diagnóstico y Análisis de Resultados}

El presente capítulo desarrolla el Objetivo Específico 2 de la investigación: \textit{``Caracterizar los patrones de fraude presentes en el dataset histórico de TechSport (gestión 2025) mediante análisis exploratorio de datos''}. Este diagnóstico valida la Hipótesis Específica 2 (HE2), que postula la existencia de al menos tres patrones de fraude recurrentes en el dataset: tarjetas robadas o clonadas, transacciones duplicadas sospechosas y comportamientos anómalos de usuarios.

El diagnóstico se estructura en cinco secciones principales: (1) caracterización del dataset de gestión 2025, (2) análisis exploratorio de datos (EDA), (3) caracterización de los patrones de fraude presentes, (4) evaluación del proceso de etiquetado de fraudes, y (5) diagnóstico de las limitaciones del sistema actual de detección.

\section{Caracterización del Dataset de Gestión 2025}

\subsection{Fuente de Datos y Población de Estudio}

La población de estudio comprende la totalidad de transacciones de pago digital procesadas por TechSport durante el año calendario 2025 (enero-diciembre). Los datos se encuentran almacenados en la base de datos operacional ClickHouse, específicamente en el esquema \texttt{TechSport\_db\_production.paybycourtDB\_payments}.

\textbf{Características cuantitativas de la población:}

\begin{itemize}[leftmargin=1.5cm]
    \item \textbf{Tamaño poblacional (N):} 15.671.512 transacciones
    \item \textbf{Período temporal:} 12 meses (01/01/2025 - 31/12/2025)
    \item \textbf{Número de variables:} 53 columnas en el esquema de base de datos
    \item \textbf{Valor monetario total:} \$3.955.095.143,24 USD
    \item \textbf{Valor promedio por transacción:} \$252,37 USD
    \item \textbf{Variable target:} Columna \texttt{is\_fraud} con etiquetas binarias (0 = legítima, 1 = fraudulenta)
\end{itemize}

\subsection{Variables Principales del Dataset}

El dataset contiene 53 variables estructuradas en las siguientes categorías:

\subsubsection{Variables de Identificación}

\begin{itemize}[leftmargin=1.5cm]
    \item \texttt{id}: Identificador único de la transacción (tipo: UUID)
    \item \texttt{user\_id}: Identificador del usuario que ejecuta la transacción (tipo: UUID)
    \item \texttt{facility\_id}: Identificador de la instalación deportiva asociada (tipo: UUID)
\end{itemize}

\subsubsection{Variables Transaccionales}

\begin{itemize}[leftmargin=1.5cm]
    \item \texttt{amount}: Monto de la transacción en USD (tipo: decimal, rango: [0.01, 50.000])
    \item \texttt{currency}: Moneda de la transacción (tipo: string, valores: USD, MXN, COP, PEN, etc.)
    \item \texttt{status}: Estado final de la transacción (tipo: string, valores: completed, failed, pending, refunded)
    \item \texttt{created\_at}: Timestamp de creación de la transacción (tipo: datetime)
    \item \texttt{updated\_at}: Timestamp de última actualización (tipo: datetime)
\end{itemize}

\subsubsection{Variables de Contexto de Pago}

\begin{itemize}[leftmargin=1.5cm]
    \item \texttt{gateway}: Pasarela de pago utilizada (tipo: string)
    \item \texttt{payment\_method}: Método de pago empleado (tipo: string, valores: card, free, reverse, cash, prepaid)
    \item \texttt{payment\_channel}: Canal de transacción (tipo: string, valores: web, mobile\_app, bank\_transfer, pos, mobile\_terminal)
    \item \texttt{card\_brand}: Marca de tarjeta si aplica (tipo: string, valores: Visa, MasterCard, American Express, Discover)
\end{itemize}

\subsubsection{Variable Target (Etiqueta de Fraude)}

\begin{itemize}[leftmargin=1.5cm]
    \item \texttt{is\_fraud}: Indicador binario de fraude (tipo: boolean/integer, valores: 0 o 1)
    \item \textbf{Fuente de etiquetado:} Equipo de contabilidad de TechSport mediante análisis post-mortem
    \item \textbf{Métodos de identificación:} (i) chargebacks confirmados por instituciones financieras, (ii) disputas resueltas como fraude, (iii) reportes de usuarios afectados verificados, (iv) revisión manual de transacciones sospechosas
    \item \textbf{Delay de etiquetado:} Entre 0 días (detección inmediata) y 5 meses (chargebacks tardíos)
\end{itemize}

\subsection{Distribución por Canal de Pago}

La Tabla \ref{tab:distribucion-canal} muestra la distribución de transacciones por canal de pago durante gestión 2025.

\begin{table}[H]
\centering
\caption{Distribución de transacciones por canal de pago (Gestión 2025)}
\label{tab:distribucion-canal}
\begin{tabular}{@{}lrr@{}}
\toprule
\textbf{Canal de Pago} & \textbf{N° Transacciones} & \textbf{Porcentaje} \\
\midrule
Web                     & 10.121.569                & 64,59\%             \\
App Móvil               & 2.010.647                 & 12,83\%             \\
Transferencia Bancaria  & 1.976.210                 & 12,61\%             \\
POS (Punto de Venta)    & 1.322.679                 & 8,44\%              \\
Terminal Móvil          & 136.407                   & 0,87\%              \\
\midrule
\textbf{Total}          & \textbf{15.671.512}       & \textbf{100,00\%}   \\
\bottomrule
\end{tabular}
\end{table}

El canal Web concentra casi dos tercios de las transacciones (64,59\%), lo cual es consistente con el modelo de negocio SaaS de TechSport donde los clubes deportivos gestionan reservas y membresías principalmente desde plataformas web administrativas. Los canales móviles (App Móvil + Terminal Móvil) representan conjuntamente 13,70\% del volumen transaccional.

\subsection{Distribución por Método de Pago}

La Tabla \ref{tab:distribucion-metodo} presenta la distribución de transacciones según el método de pago empleado.

\begin{table}[H]
\centering
\caption{Distribución de transacciones por método de pago (Gestión 2025)}
\label{tab:distribucion-metodo}
\begin{tabular}{@{}lrr@{}}
\toprule
\textbf{Método de Pago} & \textbf{N° Transacciones} & \textbf{Porcentaje} \\
\midrule
Free (Sin Cargo)        & 7.950.689                 & 50,72\%             \\
Tarjeta (Card)          & 4.090.244                 & 26,10\%             \\
Reverso (Reverse)       & 1.466.854                 & 9,36\%              \\
Efectivo (Cash)         & 816.580                   & 5,21\%              \\
Prepago (Prepaid)       & 473.239                   & 3,02\%              \\
Otros                   & 873.906                   & 5,59\%              \\
\midrule
\textbf{Total}          & \textbf{15.671.512}       & \textbf{100,00\%}   \\
\bottomrule
\end{tabular}
\end{table}

Más de la mitad de las transacciones (50,72\%) corresponden a la categoría ``Free'' (sin cargo), lo cual se explica por el modelo de negocio de TechSport donde existen transacciones de reserva que no generan cargo inmediato o están cubiertas por membresías prepagadas. Las transacciones con tarjeta (26,10\%) constituyen el segundo método más frecuente y son el principal vector de fraude financiero.

\subsection{Distribución por Gateway de Pago}

La Tabla \ref{tab:distribucion-gateway} muestra la distribución de transacciones por gateway de pago.

\begin{table}[H]
\centering
\caption{Distribución de transacciones por gateway de pago (Gestión 2025)}
\label{tab:distribucion-gateway}
\begin{tabular}{@{}lrr@{}}
\toprule
\textbf{Gateway de Pago} & \textbf{N° Transacciones} & \textbf{Porcentaje} \\
\midrule
No especificado          & 14.249.503                & 90,92\%             \\
Bolt                     & 894.847                   & 5,71\%              \\
Stripe Terminal          & 520.295                   & 3,32\%              \\
ACH                      & 6.867                     & 0,05\%              \\
\midrule
\textbf{Total}           & \textbf{15.671.512}       & \textbf{100,00\%}   \\
\bottomrule
\end{tabular}
\end{table}

La proporción de transacciones categorizadas como ``No especificado'' (90,92\%) revela una limitación significativa en la arquitectura de datos actual de TechSport. Esta categorización dificulta el análisis de desempeño de seguridad por gateway específico y representa un área de mejora en el sistema de registro transaccional.

\subsection{Distribución Temporal de Transacciones}

La Tabla \ref{tab:distribucion-temporal} presenta la distribución mensual de transacciones durante la gestión 2025, segmentada según la partición temporal definida para el modelo.

\begin{table}[H]
\centering
\caption{Distribución temporal de transacciones por conjunto de datos (Gestión 2025)}
\label{tab:distribucion-temporal}
\begin{tabular}{@{}llrr@{}}
\toprule
\textbf{Conjunto} & \textbf{Período} & \textbf{N° Transacciones} & \textbf{\% del Total} \\
\midrule
Training      & Enero-Junio 2025      & 7.835.756  & 50,00\% \\
Validation    & Julio-Agosto 2025     & 2.664.157  & 17,00\% \\
Test          & Septiembre-Dic 2025   & 5.171.599  & 33,00\% \\
\midrule
\textbf{Total} & \textbf{Gestión 2025} & \textbf{15.671.512} & \textbf{100,00\%} \\
\bottomrule
\end{tabular}
\end{table}

La división temporal en períodos consecutivos y no solapados garantiza que el modelo será evaluado en datos futuros no vistos durante el entrenamiento, evitando data leakage y simulando condiciones reales de despliegue en producción.

\section{Análisis Exploratorio de Datos (EDA)}

El Análisis Exploratorio de Datos (EDA), técnica fundamental en investigación cuantitativa \parencite{Hernandez2018}, permite comprender la estructura, distribución y características del dataset antes de desarrollar modelos predictivos.

\subsection{Estadísticas Descriptivas del Dataset}

La Tabla \ref{tab:estadisticas-amount} presenta las estadísticas descriptivas de la variable cuantitativa principal del dataset: monto de transacción (\texttt{amount}).

\begin{table}[H]
\centering
\caption{Estadísticas descriptivas de la variable \texttt{amount} (monto en USD)}
\label{tab:estadisticas-amount}
\begin{tabular}{@{}lr@{}}
\toprule
\textbf{Estadístico} & \textbf{Valor (USD)} \\
\midrule
N (transacciones)    & 15.671.512           \\
Media ($\bar{x}$)    & [TAREA POR DESARROLLAR]      \\
Mediana (Q2)         & [TAREA POR DESARROLLAR]      \\
Desviación estándar ($\sigma$) & [TAREA POR DESARROLLAR] \\
Mínimo               & [TAREA POR DESARROLLAR]      \\
Máximo               & [TAREA POR DESARROLLAR]      \\
Q1 (Percentil 25)    & [TAREA POR DESARROLLAR]      \\
Q3 (Percentil 75)    & [TAREA POR DESARROLLAR]      \\
Rango intercuartílico (IQR) & [TAREA POR DESARROLLAR] \\
Asimetría (Skewness) & [TAREA POR DESARROLLAR]      \\
Curtosis (Kurtosis)  & [TAREA POR DESARROLLAR]      \\
\bottomrule
\end{tabular}
\end{table}

[TAREA POR DESARROLLAR: Análisis de distribución - simetría vs. sesgo, outliers positivos, concentración de valores, curtosis]

\subsection{Análisis de Distribución de Clases (Fraude/No Fraude)}

La distribución de la variable target \texttt{is\_fraud} es fundamental para diseñar estrategias de balanceo de clases en el modelo de Machine Learning.

\begin{table}[H]
\centering
\caption{Distribución de clases en la variable target \texttt{is\_fraud}}
\label{tab:distribucion-clases}
\begin{tabular}{@{}lrr@{}}
\toprule
\textbf{Clase} & \textbf{Frecuencia Absoluta} & \textbf{Frecuencia Relativa} \\
\midrule
No Fraude (0)  & [TAREA POR DESARROLLAR]      & [TAREA POR DESARROLLAR]      \\
Fraude (1)     & [TAREA POR DESARROLLAR]      & [TAREA POR DESARROLLAR]      \\
\midrule
\textbf{Total} & \textbf{15.671.512}          & \textbf{100,00\%}            \\
\bottomrule
\end{tabular}
\end{table}

[TAREA POR DESARROLLAR: Análisis de desbalanceo de clases - calcular ratio de desbalanceo, determinar severidad, seleccionar estrategia metodológica (SMOTE o class\_weight)]

Coherente con estudios previos \parencite{Hafez2025}, se espera un desbalanceo severo (ratio > 100:1) dado que la tasa típica de fraude en pagos digitales es inferior al 1\% del volumen transaccional.

\subsection{Análisis de Correlación entre Features}

El análisis de correlación permite identificar relaciones lineales entre variables numéricas, detectar multicolinealidad y evaluar la asociación individual de cada feature con la variable target.

[TAREA POR DESARROLLAR: Matriz de correlación de Pearson entre features numéricas principales (amount, hour\_of\_day, day\_of\_week, user\_age\_days, tx\_count\_last\_24h, tx\_count\_last\_7d, avg\_amount\_user) y variable target is\_fraud. Incluir análisis de multicolinealidad y correlación con target.]

\subsection{Detección de Outliers en Variable \texttt{amount}}

La detección de valores atípicos (outliers) en la variable \texttt{amount} es crucial para comprender la distribución de montos transaccionales y su relación con fraude.

El método IQR define como outliers aquellos valores que se encuentran fuera de los límites:
\begin{itemize}[leftmargin=1.5cm]
    \item \textbf{Límite inferior:} $L_{inf} = Q1 - 1.5 \times IQR$
    \item \textbf{Límite superior:} $L_{sup} = Q3 + 1.5 \times IQR$
\end{itemize}

[TAREA POR DESARROLLAR: Tabla de detección de outliers con Q1, Q3, IQR, límites, número de outliers detectados, porcentaje del total, outliers fraudulentos vs legítimos. Análisis mediante tabla cruzada (crosstab) de asociación entre outliers y fraude.]

\subsection{Análisis Temporal de Transacciones}

El análisis de series temporales permite identificar patrones de estacionalidad, tendencias y anomalías en el volumen transaccional.

\begin{table}[H]
\centering
\caption{Distribución de transacciones por día de la semana}
\label{tab:estacionalidad-semanal}
\begin{tabular}{@{}lrr@{}}
\toprule
\textbf{Día de la Semana} & \textbf{N° Transacciones} & \textbf{\% del Total} \\
\midrule
Lunes                     & [TAREA POR DESARROLLAR]   & [TAREA POR DESARROLLAR] \\
Martes                    & [TAREA POR DESARROLLAR]   & [TAREA POR DESARROLLAR] \\
Miércoles                 & [TAREA POR DESARROLLAR]   & [TAREA POR DESARROLLAR] \\
Jueves                    & [TAREA POR DESARROLLAR]   & [TAREA POR DESARROLLAR] \\
Viernes                   & [TAREA POR DESARROLLAR]   & [TAREA POR DESARROLLAR] \\
Sábado                    & [TAREA POR DESARROLLAR]   & [TAREA POR DESARROLLAR] \\
Domingo                   & [TAREA POR DESARROLLAR]   & [TAREA POR DESARROLLAR] \\
\midrule
\textbf{Total}            & \textbf{15.671.512}       & \textbf{100,00\%}       \\
\bottomrule
\end{tabular}
\end{table}

[TAREA POR DESARROLLAR: Análisis de tendencia mediante regresión lineal, detección de picos anómalos usando método z-score]

\subsection{Tasa de Fraude por Canal de Pago}

El análisis de tasa de fraude segmentado por canal permite identificar vectores de ataque prioritarios.

\begin{table}[H]
\centering
\caption{Tasa de fraude por canal de pago (Gestión 2025)}
\label{tab:fraude-canal}
\begin{tabular}{@{}lrrrr@{}}
\toprule
\textbf{Canal} & \textbf{N° Total} & \textbf{N° Fraudes} & \textbf{Tasa Fraude} & \textbf{Pérdidas (USD)} \\
\midrule
Web                     & 10.121.569 & [TAREA] & [TAREA] & [TAREA] \\
App Móvil               & 2.010.647  & [TAREA] & [TAREA] & [TAREA] \\
Transferencia Bancaria  & 1.976.210  & [TAREA] & [TAREA] & [TAREA] \\
POS (Punto de Venta)    & 1.322.679  & [TAREA] & [TAREA] & [TAREA] \\
Terminal Móvil          & 136.407    & [TAREA] & [TAREA] & [TAREA] \\
\midrule
\textbf{Total}          & \textbf{15.671.512} & [TAREA] & [TAREA] & [TAREA] \\
\bottomrule
\end{tabular}
\end{table}

[TAREA POR DESARROLLAR: Interpretación de resultados - canal más vulnerable, canal con mayores pérdidas absolutas, recomendaciones de priorización]

\subsection{Análisis de Valores Faltantes (Missing Values)}

La presencia de valores faltantes puede afectar el desempeño del modelo de Machine Learning.

[TAREA POR DESARROLLAR: Tabla de análisis de valores faltantes en variables críticas (amount, gateway, payment\_method, payment\_channel, card\_brand, user\_id, facility\_id, created\_at) con número de missing, porcentaje y estrategia de tratamiento]

\subsection{Análisis de Transacciones Duplicadas}

La detección de transacciones duplicadas es crítica dado que constituye uno de los tres patrones de fraude objetivo de la investigación (Patrón 2).

Una transacción se considera duplicada si coincide con otra en: mismo \texttt{user\_id}, mismo \texttt{amount} (± \$0,01), mismo \texttt{facility\_id}, y timestamp dentro de ventana de 5 minutos.

[TAREA POR DESARROLLAR: Tabla de análisis de duplicados con transacciones únicas vs duplicadas, duplicados fraudulentos vs legítimos, análisis de naturaleza de duplicados]

\subsection{Feature Importance Preliminar (Análisis Univariado)}

El análisis univariado de importancia de features permite identificar qué variables individuales tienen mayor asociación con la variable target \texttt{is\_fraud}.

[TAREA POR DESARROLLAR: Tabla de top 15 features con mayor asociación univariada - nombre, tipo, correlación/Chi2, p-value. Interpretación y selección de features candidatas para Random Forest]

\section{Caracterización de Patrones de Fraude}

Esta sección desarrolla la caracterización de los tres principales patrones de fraude identificados en el dataset de TechSport, validando la Hipótesis Específica 2 (HE2).

\subsection{Patrón 1: Uso de Tarjetas Robadas o Clonadas}

El patrón de tarjetas robadas o clonadas se caracteriza por el uso no autorizado de credenciales de pago obtenidas ilícitamente (mediante phishing, skimming de cajeros automáticos, brechas de seguridad en comercios, o compra en mercados clandestinos).

\textbf{Indicadores técnicos característicos:}

\begin{enumerate}[leftmargin=1.5cm]
    \item \textbf{Múltiples tarjetas desde misma dirección IP:} Múltiples transacciones utilizando diferentes números de tarjeta desde la misma IP en ventana temporal < 1 hora.

    \item \textbf{Transacciones de alto monto seguidas de chargeback:} Transacciones con monto > percentil 90 del usuario que resultan en chargeback confirmado.

    \item \textbf{Mismatch geográfico de tarjeta:} Inconsistencia entre país de emisión de tarjeta y país de origen de la transacción.

    \item \textbf{Velocidad transaccional anómala:} Múltiples intentos de transacción en secuencia rápida (< 30 segundos entre intentos).

    \item \textbf{Primera transacción de alto valor:} Nueva tarjeta que ejecuta inmediatamente transacción de monto > \$500 sin historial previo.
\end{enumerate}

\begin{table}[H]
\centering
\caption{Caracterización cuantitativa del Patrón 1 (Gestión 2025)}
\label{tab:patron1-cuantitativo}
\begin{tabular}{@{}lr@{}}
\toprule
\textbf{Métrica} & \textbf{Valor} \\
\midrule
N° casos detectados              & [TAREA POR DESARROLLAR] \\
\% del total de fraudes          & [TAREA POR DESARROLLAR] \\
Monto promedio por caso          & [TAREA POR DESARROLLAR] USD \\
Monto total de pérdidas          & [TAREA POR DESARROLLAR] USD \\
Canal más afectado               & [TAREA POR DESARROLLAR] \\
Gateway más afectado             & [TAREA POR DESARROLLAR] \\
Mes con mayor incidencia         & [TAREA POR DESARROLLAR] \\
\bottomrule
\end{tabular}
\end{table}

\subsection{Patrón 2: Transacciones Duplicadas Sospechosas}

El patrón de transacciones duplicadas sospechosas se caracteriza por la ejecución de múltiples transacciones prácticamente idénticas por el mismo usuario en una ventana temporal no justificada por el modelo de negocio.

\textbf{Criterio técnico de detección:}

Una transacción se clasifica como duplicado sospechoso si cumple simultáneamente:
\begin{itemize}[leftmargin=1.5cm]
    \item Mismo \texttt{user\_id}
    \item Mismo \texttt{facility\_id}
    \item Monto idéntico o con variación < 1\%
    \item Timestamp separados por < 5 minutos
    \item Exclusión de casos legítimos: transacciones recurrentes programadas
\end{itemize}

\begin{table}[H]
\centering
\caption{Caracterización cuantitativa del Patrón 2 (Gestión 2025)}
\label{tab:patron2-cuantitativo}
\begin{tabular}{@{}lr@{}}
\toprule
\textbf{Métrica} & \textbf{Valor} \\
\midrule
N° casos detectados              & [TAREA POR DESARROLLAR] \\
\% del total de fraudes          & [TAREA POR DESARROLLAR] \\
Monto promedio por caso          & [TAREA POR DESARROLLAR] USD \\
Monto total de pérdidas          & [TAREA POR DESARROLLAR] USD \\
Canal más afectado               & [TAREA POR DESARROLLAR] \\
Tiempo promedio entre duplicados & [TAREA POR DESARROLLAR] segundos \\
\bottomrule
\end{tabular}
\end{table}

\subsection{Patrón 3: Comportamientos Anómalos de Usuarios}

El patrón de comportamientos anómalos se caracteriza por desviaciones significativas respecto al perfil histórico de comportamiento transaccional del usuario.

\textbf{Indicadores técnicos característicos:}

\begin{enumerate}[leftmargin=1.5cm]
    \item \textbf{Anomalía en monto transaccional (z-score):} $z\_score = \frac{amount - \mu_{user}}{\sigma_{user}}$. Criterio de anomalía: $|z\_score| > 3$.

    \item \textbf{Velocidad transaccional anómala:} \texttt{tx\_count\_last\_24h} > 5 × promedio histórico diario del usuario.

    \item \textbf{Cambio geográfico abrupto:} Cambio de país de IP sin historial previo de transacciones internacionales.

    \item \textbf{Cambio de dispositivo/navegador inusual:} Usuario ejecuta transacción desde dispositivo completamente distinto sin período de transición.

    \item \textbf{Horario de actividad anómalo:} Transacciones en horario atípico para el usuario (madrugada si historial es diurno).
\end{enumerate}

\begin{table}[H]
\centering
\caption{Caracterización cuantitativa del Patrón 3 (Gestión 2025)}
\label{tab:patron3-cuantitativo}
\begin{tabular}{@{}lr@{}}
\toprule
\textbf{Métrica} & \textbf{Valor} \\
\midrule
N° casos detectados              & [TAREA POR DESARROLLAR] \\
\% del total de fraudes          & [TAREA POR DESARROLLAR] \\
Monto promedio por caso          & [TAREA POR DESARROLLAR] USD \\
Monto total de pérdidas          & [TAREA POR DESARROLLAR] USD \\
\midrule
\multicolumn{2}{l}{\textbf{Subtipos de anomalía:}} \\
Anomalía de monto (z-score > 3)      & [TAREA POR DESARROLLAR] \\
Anomalía de velocidad transaccional  & [TAREA POR DESARROLLAR] \\
Anomalía geográfica (IP country)     & [TAREA POR DESARROLLAR] \\
Cambio de dispositivo sospechoso     & [TAREA POR DESARROLLAR] \\
Horario anómalo                      & [TAREA POR DESARROLLAR] \\
\bottomrule
\end{tabular}
\end{table}

\subsection{Distribución de Patrones de Fraude}

La Tabla \ref{tab:distribucion-patrones} presenta la distribución comparativa de los tres patrones de fraude.

\begin{table}[H]
\centering
\caption{Distribución comparativa de patrones de fraude (Gestión 2025)}
\label{tab:distribucion-patrones}
\begin{tabular}{@{}lrrrr@{}}
\toprule
\textbf{Patrón} & \textbf{N° Casos} & \textbf{\% Fraudes} & \textbf{Monto Prom.} & \textbf{Canal Ppal.} \\
\midrule
Patrón 1: Tarjetas robadas    & [TAREA] & [TAREA] & [TAREA] & [TAREA] \\
Patrón 2: Duplicados sosp.    & [TAREA] & [TAREA] & [TAREA] & [TAREA] \\
Patrón 3: Comportamientos an. & [TAREA] & [TAREA] & [TAREA] & [TAREA] \\
\midrule
\textbf{Total fraudes} & [TAREA POR DESARROLLAR] & \textbf{100\%} & [TAREA] & --- \\
\bottomrule
\end{tabular}
\end{table}

[TAREA POR DESARROLLAR: Análisis comparativo de patrones - patrón dominante, patrón de mayor impacto económico, canales de mayor vulnerabilidad por patrón, solapamiento de patrones]

\section{Evaluación del Proceso de Etiquetado de Fraudes}

La confiabilidad de las etiquetas de fraude (variable \texttt{is\_fraud}) es crítica para el entrenamiento supervisado del modelo de Machine Learning.

\subsection{Fuentes de Etiquetado de Fraude}

[TAREA POR DESARROLLAR: Tabla con fuentes de etiquetado - chargebacks confirmados, disputas resueltas, reportes de usuarios, revisión manual - con número de fraudes detectados por cada fuente y porcentaje]

\subsection{Análisis de Delay de Etiquetado}

[TAREA POR DESARROLLAR: Histograma de distribución del tiempo entre transacción y etiquetado (0-5 meses), estadísticos de delay, tabla de rangos de delay con número de fraudes por rango]

\subsection{Consistencia Temporal del Etiquetado}

[TAREA POR DESARROLLAR: Tabla de tasa de fraude mensual para detectar inconsistencias sistemáticas, cálculo de media y desviación estándar, verificación de outliers]

\section{Diagnóstico del Sistema Actual de Detección de Fraude}

Esta sección desarrolla el diagnóstico crítico del sistema actual de detección de fraude de TechSport, identificando sus limitaciones operacionales y técnicas.

\subsection{Descripción del Sistema Actual}

[TAREA POR DESARROLLAR: Descripción de la arquitectura del sistema actual basado en reglas estáticas, ejemplos de reglas implementadas, proceso de actualización de reglas, responsables]

\subsection{Limitaciones Identificadas del Sistema Actual}

\subsubsection{Limitación 1: Detección Post-Mortem de Fraudes}

[TAREA POR DESARROLLAR: Evidencia cuantitativa del porcentaje de fraudes detectados mediante chargebacks tardíos, consecuencias, comparación con sistema proactivo]

\subsubsection{Limitación 2: Actualización Manual Constante de Reglas}

[TAREA POR DESARROLLAR: Frecuencia de actualización de reglas, problema de evolución de patrones, ventana de vulnerabilidad]

\subsubsection{Limitación 3: Ausencia de Correlación Cruzada entre Gateways}

[TAREA POR DESARROLLAR: Problema de falta de correlación, ejemplo de patrón cruzado no detectado, consecuencias]

\subsubsection{Limitación 4: Alta Tasa de Falsos Positivos}

[TAREA POR DESARROLLAR: Porcentaje de transacciones legítimas bloqueadas incorrectamente, cálculo de pérdidas por falsos positivos, impacto en experiencia del usuario]

\subsection{Desempeño del Sistema Actual (Baseline)}

[TAREA POR DESARROLLAR: Si existen logs de alertas del sistema actual, calcular métricas baseline - Precision, Recall, F1-Score. Estos valores serán benchmark para comparar con el modelo ML en Capítulo 3]

\section{Síntesis del Diagnóstico}

Esta sección integra los hallazgos de las secciones anteriores, respondiendo directamente al Objetivo Específico 2 y validando la Hipótesis Específica 2.

\subsection{Hallazgos Principales del Diagnóstico}

\begin{enumerate}[leftmargin=1.5cm]
    \item \textbf{Dataset robusto disponible:} Gestión 2025 comprende 15.671.512 transacciones con 53 variables, valor monetario total de \$3.955M USD, y variable target \texttt{is\_fraud} validada por equipo de contabilidad.

    \item \textbf{Desbalanceo de clases confirmado:} [TAREA POR DESARROLLAR - ratio exacto]. La tasa de fraude requiere estrategias de balanceo (SMOTE o class\_weight) para entrenamiento del modelo ML.

    \item \textbf{Tres patrones de fraude caracterizados:} (i) tarjetas robadas/clonadas, (ii) transacciones duplicadas sospechosas, (iii) comportamientos anómalos de usuarios. Cada patrón presenta características técnicas específicas identificables mediante features comportamentales.

    \item \textbf{Proceso de etiquetado validado:} El equipo de contabilidad utiliza 4 fuentes de verificación (chargebacks, disputas, reportes, revisión manual) con delay promedio de [TAREA POR DESARROLLAR] días.

    \item \textbf{Sistema actual con limitaciones críticas:} Detección post-mortem, ausencia de aprendizaje automático, incapacidad de correlación cruzada multigateway, y alta tasa de falsos positivos.
\end{enumerate}

\subsection{Validación de Hipótesis Específica 2 (HE2)}

La Hipótesis Específica 2 establece: \textit{``El análisis exploratorio del dataset de TechSport revela al menos 3 patrones de fraude recurrentes: tarjetas robadas/clonadas, transacciones duplicadas sospechosas, y comportamientos anómalos de usuarios''}.

[TAREA POR DESARROLLAR: Con base en los datos reales del EDA, validar explícitamente si se confirma o rechaza HE2, indicando el número y porcentaje de cada patrón identificado]

\subsection{Justificación de la Necesidad del Modelo ML}

Los hallazgos del diagnóstico demuestran que:

\begin{itemize}[leftmargin=1.5cm]
    \item El dataset de TechSport gestión 2025 cumple con los requisitos cuantitativos y cualitativos para entrenar un modelo de Machine Learning supervisado.

    \item Los tres patrones de fraude identificados presentan características medibles y correlacionadas que un modelo Random Forest puede aprender mediante análisis de 15+ features comportamentales.

    \item El sistema actual basado en reglas estáticas es insuficiente para la detección proactiva de fraude, justificando la implementación de un modelo inteligente.

    \item El diagnóstico confirma la viabilidad de alcanzar las métricas objetivo establecidas en la Hipótesis General, dado que estudios previos \parencite{Hafez2025} reportan F1-Scores de 85-94\% en contextos similares.
\end{itemize}

\subsection{Transición al Capítulo 3}

El Capítulo 2 ha diagnosticado la situación actual del sistema de detección de fraude de TechSport, caracterizando el dataset de gestión 2025, identificando los tres patrones de fraude presentes, y documentando las limitaciones del sistema basado en reglas estáticas. El Capítulo 3 desarrollará la propuesta de solución mediante la implementación del modelo de Machine Learning supervisado basado en Random Forest.

\cleardoublepage
