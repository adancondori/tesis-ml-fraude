% ==================================================================================
% FUNDAMENTACIÓN DE INSTRUMENTOS Y PROCEDIMIENTO DIAGNÓSTICO
% ==================================================================================
\documentclass[12pt,a4paper]{article}

% Paquetes básicos
\usepackage[utf8]{inputenc}
\usepackage[T1]{fontenc}
\usepackage[spanish,es-tabla]{babel}
\usepackage{geometry}
\geometry{left=3cm,right=2.5cm,top=2.5cm,bottom=2.5cm}
\usepackage{setspace}
\onehalfspacing
\usepackage{enumitem}
\usepackage{hyperref}
\usepackage{csquotes}

\begin{document}

\begin{center}
    \textbf{\Large Fundamentación de Instrumentos y Procedimiento Diagnóstico}\\[0.5em]
    \textbf{Implementación de un Modelo de Machine Learning para la Detección de Anomalías y Fraude en Pagos Transaccionales en TechSport (2024--2025)}
\end{center}

\section*{1. Propósito y marco metodológico}
El diagnóstico se orienta por el método científico planteado por Hernández Sampieri (2014), asegurando la coherencia entre problema, objetivos y medición de variables. Tal como se estableció en la \textit{Tarea 2 -- Selección de métodos del nivel teórico}, se articulan los razonamientos analítico-sintético, inductivo-deductivo, de modelación e hipotético-deductivo. El corpus conceptual compilado en la \textit{Tarea 3 -- Fundamentos teóricos referenciales} proporciona el soporte epistemológico que permite operacionalizar cada instrumento dentro de las fases de observación, experimentación, contrastación y retroalimentación descritas por Sampieri.

\section*{2. Fundamentación de los instrumentos seleccionados}
\begin{itemize}[leftmargin=*]
    \item \textbf{Observación técnica directa y revisión de logs.} Siguiendo los lineamientos de observación estructurada de Sampieri, se analizan los logs transaccionales y reportes de fraude documentados en \texttt{capitulos/03\_metodologia.tex:88-93}. Este instrumento caracteriza el desempeño del sistema actual, identifica patrones de error y establece la línea base que el nuevo modelo debe superar, aportando validez externa al vincular el marco teórico con la realidad operativa de TechSport.
    \item \textbf{Dataset anonimizado de transacciones.} Constituye el instrumento central para las variables independiente y dependiente, conforme a las definiciones y operacionalización descritas en \texttt{capitulos/03\_metodologia.tex:31-86}. Incluye atributos (monto, país, canal, pasarela, \textit{timestamp}) y etiquetas de clasificación, cumpliendo con el criterio de Sampieri de emplear instrumentos estandarizados que aseguren confiabilidad dentro de la delimitación temporal y espacial del estudio.
    \item \textbf{Guía de evaluación del modelo.} Derivada de la matriz de indicadores (precisión, \textit{recall}, F1-score, tasa de falsos positivos, tiempo de respuesta), formaliza los criterios de comparación entre el modelo inteligente y el sistema basado en reglas. Opera como instrumento de control de calidad que incrementa la validez interna del experimento comparativo e integra los aportes de la \textit{Tarea 3} sobre métricas y estándares de ciberseguridad (NIST CSF 2.0, PCI DSS).
    \item \textbf{Scripts de procesamiento y análisis.} Los cuadernos en Python (Pandas, Scikit-learn, Matplotlib) garantizan trazabilidad y replicabilidad del análisis. Permiten ejecutar validación cruzada k-fold, generar matrices de confusión y curvas ROC, conforme al principio sampieriano de documentar procedimientos para sostener el método hipotético-deductivo.
\end{itemize}

\section*{3. Procedimiento diagnóstico alineado al método científico}
\begin{enumerate}[leftmargin=*]
    \item \textbf{Planeación conceptual.} Integrar antecedentes, variables y objetivos recogidos en el perfil (\texttt{capitulos/00\_perfil.tex}) y en la \textit{Tarea 3} para formular hipótesis diagnósticas sobre las causas del bajo desempeño actual.
    \item \textbf{Levantamiento de información.} Ejecutar la observación técnica directa mediante extracción de logs y solicitar al equipo de TechSport el dataset anonimizado siguiendo los criterios éticos definidos en la delimitación temporal y espacial (\texttt{capitulos/03\_metodologia.tex:14-28}).
    \item \textbf{Preparación y estructuración de datos.} Aplicar los scripts de limpieza, balanceo y partición 70/15/15 establecidos en la metodología, registrando cada transformación en una bitácora técnica para preservar la confiabilidad del instrumento.
    \item \textbf{Experimentación y contraste.} Entrenar los modelos propuestos, ejecutar validación cruzada k-fold (k=5) y comparar resultados con el sistema actual mediante la guía de evaluación. Documentar métricas en tablas y visualizaciones reproducibles.
    \item \textbf{Síntesis y retroalimentación.} Elaborar el informe diagnóstico que articule hallazgos cuantitativos con el análisis teórico de las Tareas 2 y 3, identificando brechas, riesgos y oportunidades de mejora. Incorporar recomendaciones alineadas a los estándares NIST CSF y PCI DSS.
\end{enumerate}

\section*{4. Garantías de validez y confiabilidad}
\begin{itemize}[leftmargin=*]
    \item \textbf{Validez interna.} La guía de evaluación y la comparación controlada con el sistema basado en reglas permiten aislar el efecto del modelo de Machine Learning, cumpliendo la lógica experimental de Sampieri.
    \item \textbf{Validez externa.} El uso de datos reales de TechSport y la observación directa de logs aseguran que los hallazgos sean transferibles al entorno operativo, reforzando la pertinencia del diagnóstico frente a los objetivos institucionales.
    \item \textbf{Confiabilidad.} La estandarización del dataset, el versionamiento de scripts y la bitácora de transformaciones facilitan la repetición del estudio y la auditoría de resultados, en concordancia con la recomendación de Sampieri sobre instrumentos reproducibles.
\end{itemize}

\section*{5. Articulación con entregables previos y próximos pasos}
\begin{itemize}[leftmargin=*]
    \item La matriz de métodos y la estructura conceptual trabajadas en la \textit{Tarea 2} respaldan la selección de instrumentos al demostrar su alineación con los objetivos específicos y la hipótesis de trabajo.
    \item El compendio bibliográfico y la matriz de referencias de la \textit{Tarea 3} justifican la elección de métricas, técnicas de aprendizaje y estándares de ciberseguridad considerados en la guía de evaluación.
    \item El material de la \textit{Tarea 4} (marco contextual) refuerza la necesidad de situar el diagnóstico en las características institucionales y tecnológicas de TechSport, evitando sesgos por información periférica.
\end{itemize}

\noindent\textbf{Próximos pasos inmediatos:} consolidar el dataset anonimizado con metadatos completos, formalizar la guía de evaluación en formato \textit{checklist} y preparar el primer informe de observación basado en los logs históricos para presentar al tutor.

\end{document}
