% ==================================================================================
% PAQUETES NECESARIOS PARA LA TESIS
% ==================================================================================

% Codificación y fuentes
\usepackage[utf8]{inputenc}
\usepackage[T1]{fontenc}
\usepackage{lmodern}

% Idioma español
\usepackage[spanish,es-tabla,es-nodecimaldot]{babel}

% Sangría en el primer párrafo después de secciones
\usepackage{indentfirst}

% Geometría de la página
\usepackage[
    left=3cm,
    right=2.5cm,
    top=2.5cm,
    bottom=2.5cm,
    headheight=16pt
]{geometry}

% Interlineado
\usepackage{setspace}
\onehalfspacing  % 1.5 de interlineado

% Enlaces e hipervínculos
\usepackage[hidelinks,colorlinks=true,linkcolor=black,citecolor=blue,urlcolor=blue]{hyperref}
\usepackage{url}

% Gráficos y figuras
\usepackage{graphicx}
\usepackage{float}
\usepackage{caption}
\usepackage{subcaption}
\usepackage{wrapfig}

% Ruta de las imágenes
\graphicspath{{imagenes/}{imagenes/figuras/}{imagenes/diagramas/}{imagenes/graficos/}{imagenes/logos/}}

% Tablas
\usepackage{booktabs}
\usepackage{multirow}
\usepackage{longtable}
\usepackage{array}
\usepackage{tabularx}
\usepackage{colortbl}

% Matemáticas
\usepackage{amsmath}
\usepackage{amssymb}
\usepackage{amsfonts}
\usepackage{amsthm}
\usepackage{mathtools}

% Bibliografía (estilo APA 7ma edición)
\usepackage[
    backend=biber,
    style=apa,
    sorting=nyt,
    maxbibnames=99,
    uniquename=false,
    uniquelist=false,
    apamaxprtauth=99
]{biblatex}
\DeclareLanguageMapping{spanish}{spanish-apa}
\addbibresource{bibliografia/referencias.bib}

% Código fuente
\usepackage{listings}
\usepackage{xcolor}

% Algoritmos
\usepackage[ruled,vlined,linesnumbered]{algorithm2e}

% Encabezados y pies de página
\usepackage{fancyhdr}

% Formato de títulos
\usepackage{titlesec}

% Índice de contenidos personalizado
\usepackage{tocloft}

% Apéndices
\usepackage[toc,page]{appendix}

% Glosarios y acrónimos
\usepackage[acronym,toc]{glossaries}

% Herramientas adicionales
\usepackage{etoolbox}
\usepackage{xifthen}
\usepackage{ifmtarg}
\usepackage{enumitem}

% Formato de fechas
\usepackage[spanish]{datetime}

% Química y ecuaciones (si es necesario)
% \usepackage{chemfig}
% \usepackage{siunitx}

% PDFs y elementos externos
\usepackage{pdfpages}

% Diagramas y gráficos (TikZ)
\usepackage{tikz}
\usetikzlibrary{shapes,arrows,positioning,calc}

% Código de colores personalizado
\definecolor{codegreen}{rgb}{0,0.6,0}
\definecolor{codegray}{rgb}{0.5,0.5,0.5}
\definecolor{codepurple}{rgb}{0.58,0,0.82}
\definecolor{backcolour}{rgb}{0.95,0.95,0.92}

% Quotes y citas
\usepackage{csquotes}

% Mejoras tipográficas
\usepackage{microtype}
