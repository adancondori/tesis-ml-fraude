% ==================================================================================
% FORMATO Y ESTILOS DEL DOCUMENTO
% ==================================================================================

% ==================== ESTILO DE ENCABEZADOS Y PIES DE PÁGINA ====================
\pagestyle{fancy}
\fancyhf{}
\fancyhead[R]{\thepage}
\fancyhead[L]{\leftmark}
\renewcommand{\headrulewidth}{0.4pt}
\renewcommand{\footrulewidth}{0pt}

% Estilo para páginas de capítulo
\fancypagestyle{plain}{
    \fancyhf{}
    \fancyfoot[C]{\thepage}
    \renewcommand{\headrulewidth}{0pt}
}

% ==================== FORMATO DE TÍTULOS ====================
% Capítulos - Centrado y en MAYÚSCULAS en UNA SOLA LÍNEA (requisito UAGRM)
\titleformat{\chapter}[block]
    {\normalfont\Large\bfseries\centering}
    {\MakeUppercase{\chaptertitlename\ \thechapter.}}
    {0.5em}
    {\MakeUppercase}
\titlespacing*{\chapter}{0pt}{-20pt}{40pt}

% Secciones
\titleformat{\section}
    {\normalfont\Large\bfseries}
    {\thesection}
    {1em}
    {}
\titlespacing*{\section}{0pt}{3.5ex plus 1ex minus .2ex}{2.3ex plus .2ex}

% Subsecciones
\titleformat{\subsection}
    {\normalfont\large\bfseries}
    {\thesubsection}
    {1em}
    {}
\titlespacing*{\subsection}{0pt}{3.25ex plus 1ex minus .2ex}{1.5ex plus .2ex}

% ==================== FORMATO DE LISTINGS (CÓDIGO) ====================
\lstdefinestyle{mystyle}{
    backgroundcolor=\color{backcolour},
    commentstyle=\color{codegreen},
    keywordstyle=\color{magenta},
    numberstyle=\tiny\color{codegray},
    stringstyle=\color{codepurple},
    basicstyle=\ttfamily\footnotesize,
    breakatwhitespace=false,
    breaklines=true,
    captionpos=b,
    keepspaces=true,
    numbers=left,
    numbersep=5pt,
    showspaces=false,
    showstringspaces=false,
    showtabs=false,
    tabsize=2,
    frame=single,
    rulecolor=\color{black}
}

\lstset{style=mystyle, inputencoding=utf8, extendedchars=true}

% Configuración específica para Python
\lstdefinestyle{python}{
    language=Python,
    style=mystyle,
    morekeywords={self,True,False,None},
    literate={á}{{\'a}}1 {é}{{\'e}}1 {í}{{\'i}}1 {ó}{{\'o}}1 {ú}{{\'u}}1
             {Á}{{\'A}}1 {É}{{\'E}}1 {Í}{{\'I}}1 {Ó}{{\'O}}1 {Ú}{{\'U}}1
             {ñ}{{\~n}}1 {Ñ}{{\~N}}1 {ü}{{\"u}}1 {Ü}{{\"U}}1,
    inputencoding=utf8,
    extendedchars=true
}

% ==================== FORMATO DE ALGORITMOS ====================
% Comentadas porque ya no usamos algpseudocode, ahora usamos algorithm2e
% \renewcommand{\algorithmicrequire}{\textbf{Entrada:}}
% \renewcommand{\algorithmicensure}{\textbf{Salida:}}
% \renewcommand{\algorithmiccomment}[1]{// #1}

% ==================== FORMATO DE TEOREMAS Y DEFINICIONES ====================
\theoremstyle{definition}
\newtheorem{definition}{Definición}[chapter]
\newtheorem{theorem}{Teorema}[chapter]
\newtheorem{corollary}{Corolario}[theorem]
\newtheorem{lemma}[theorem]{Lema}
\newtheorem{example}{Ejemplo}[chapter]
\newtheorem{remark}{Observación}[chapter]

% ==================== FORMATO DE CAPTIONS (NORMAS UAGRM) ====================
% Configuración global de captions
\captionsetup{
    font={stretch=1.5},       % Interlineado 1.5 dentro de captions
    labelfont=bf,             % Etiqueta en negrita
    format=plain,
    justification=centering
}

% FIGURAS: Caption ARRIBA según UAGRM
\captionsetup[figure]{
    name=Figura,
    labelsep=period,
    position=top,             % Caption arriba (requisito UAGRM)
    font=normalsize           % Tamaño 12pt
}

% TABLAS: Caption ARRIBA según UAGRM
\captionsetup[table]{
    name=Tabla,
    labelsep=period,
    position=top,             % Caption arriba (requisito UAGRM)
    font=normalsize           % Tamaño 12pt
}

% ==================== COMANDOS REUTILIZABLES DRY ====================
% Comando para agregar FUENTE debajo de tablas/figuras (requisito UAGRM)
% Uso: \fuente{Elaboración propia} o \fuente{Autor, año}
\newcommand{\fuente}[1]{%
    \vspace{-0.3cm}%
    \begin{flushleft}%
        \small\textit{Fuente:} #1%
    \end{flushleft}%
}

% Comando para interlineado específico en tablas (UAGRM: 1.5)
\newcommand{\tablainterlineado}{\setstretch{1.5}}

% Atajo para tablas pequeñas (tamaño 10pt según UAGRM)
\newcommand{\iniciotabla}{\small\tablainterlineado}
\newcommand{\fintabla}{\normalsize}

% ==================== FORMATO DEL ÍNDICE ====================
\renewcommand{\contentsname}{Índice General}
\renewcommand{\listfigurename}{Índice de Figuras}
\renewcommand{\listtablename}{Índice de Tablas}
\renewcommand{\bibname}{Referencias Bibliográficas}
\renewcommand{\appendixname}{Apéndice}
\renewcommand{\appendixpagename}{Apéndices}
\renewcommand{\appendixtocname}{Apéndices}

% Capítulos en MAYÚSCULAS en el índice (UAGRM)
% Inicia con "CAPÍTULO", luego main.tex cambia a "APÉNDICE" con \addtocontents
\renewcommand{\cftchappresnum}{CAPÍTULO\ }
\renewcommand{\cftchapaftersnum}{.}
\renewcommand{\cftchapfont}{\bfseries\MakeUppercase}
\renewcommand{\cftchappagefont}{\bfseries}
\setlength{\cftchapnumwidth}{7.5em}  % Espacio para "CAPÍTULO X." o "APÉNDICE A."
\renewcommand{\cftdot}{.}  % Puntos guía
\renewcommand{\cftchapleader}{\cftdotfill{\cftdotsep}}  % Puntos hasta número de página

% Espaciado en el índice
\setlength{\cftbeforechapskip}{0.5em}
\setlength{\cftbeforesecskip}{0.3em}

% ==================== ESPACIADO Y MÁRGENES ====================
% Sangría en párrafos (estilo APA: 0.5 pulgadas = 1.27cm)
\setlength{\parindent}{1.27cm}
% Espacio entre párrafos
\setlength{\parskip}{0pt plus 0.5ex}

% ==================== CONFIGURACIÓN DE ENUMERACIONES ====================
\setlist[itemize]{noitemsep, topsep=0pt}
\setlist[enumerate]{noitemsep, topsep=0pt}

% ==================== CONFIGURACIÓN DE HIPERVÍNCULOS ====================
\hypersetup{
    pdftitle={Implementación de un Modelo de Machine Learning para Detección de Fraude},
    pdfauthor={Tu Nombre},
    pdfsubject={Tesis de Maestría},
    pdfkeywords={Machine Learning, Fraud Detection, Anomaly Detection, PaymentSystems},
    bookmarksnumbered=true,
    bookmarksopen=true,
    bookmarksopenlevel=1,
    pdfstartview=Fit,
    pdfpagemode=UseOutlines
}
