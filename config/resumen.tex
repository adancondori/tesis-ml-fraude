% ==================================================================================
% RESUMEN
% ==================================================================================

\chapter*{Resumen}
\addcontentsline{toc}{chapter}{Resumen}

% En APA 7, el abstract NO lleva sangría en ningún párrafo
% Usamos un grupo {} para que el cambio sea local
{
\setlength{\parindent}{0pt}

La detección de fraude en los pagos digitales representa uno de los desafíos más críticos en la economía digital contemporánea, donde las transacciones electrónicas experimentan un crecimiento exponencial y las técnicas fraudulentas evolucionan constantemente. Esta investigación propone la implementación de un modelo de Machine Learning supervisado para la detección de anomalías y fraude en pagos transaccionales en la empresa TechSport, ubicada en Miami, Florida, durante la gestión 2024-2025.

El estudio adopta un enfoque cuantitativo, de tipo aplicado y diseño experimental-comparativo, analizando datos históricos de transacciones procesadas a través de múltiples pasarelas de pago (Stripe, CardConnect, Kushki, entre otras) y diversos canales (web, aplicación móvil y puntos de venta). La investigación se enmarca en el área de Sistemas Inteligentes, específicamente en Sistemas Cognitivos, contribuyendo al cuerpo de conocimientos sobre aplicación de inteligencia artificial en seguridad financiera.

La metodología incluye la recopilación y preprocesamiento de datos transaccionales, el entrenamiento de modelos supervisados utilizando algoritmos de clasificación, y la validación mediante métricas estándar como precisión, recall, F1-score y tasa de falsos positivos. Se implementa validación cruzada k-fold (k=5) para garantizar la robustez del modelo y se compara el desempeño del sistema propuesto con el método actual basado en reglas estáticas.

Los resultados demuestran que el modelo de Machine Learning implementado supera significativamente al sistema tradicional en términos de capacidad de detección, reducción de falsos positivos y adaptabilidad ante nuevas modalidades de fraude. El modelo alcanza métricas superiores al 94\% de precisión en la identificación de transacciones fraudulentas, manteniendo una tasa de falsos positivos inferior al 5\%.

Esta investigación contribuye al campo académico proporcionando evidencia empírica sobre la efectividad de modelos supervisados en contextos empresariales reales, y aporta valor práctico al sector fintech mediante una solución escalable y replicable en plataformas con arquitecturas similares. Asimismo, sienta las bases para futuras mejoras tecnológicas e integraciones más avanzadas en sistemas de detección de fraude.

\vspace{1cm}

\noindent\textbf{Palabras clave:} \palabrasclave
} % Fin del grupo sin sangría

\cleardoublepage
