% ==================================================================================
% COMANDOS PERSONALIZADOS PARA LA TESIS
% ==================================================================================

% ==================== COMANDOS MATEMÁTICOS ====================
% Vectores
\newcommand{\vect}[1]{\boldsymbol{#1}}

% Matrices
\newcommand{\mat}[1]{\mathbf{#1}}

% Números reales
\newcommand{\R}{\mathbb{R}}

% Números naturales
\newcommand{\N}{\mathbb{N}}

% Probabilidad
\newcommand{\Prob}[1]{\mathbb{P}\left(#1\right)}

% Esperanza
\newcommand{\E}[1]{\mathbb{E}\left[#1\right]}

% Varianza
\newcommand{\Var}[1]{\text{Var}\left(#1\right)}

% ==================== COMANDOS PARA CÓDIGO ====================
% Código inline
\newcommand{\code}[1]{\texttt{#1}}

% Resaltar términos importantes
\newcommand{\term}[1]{\textbf{\textit{#1}}}

% ==================== COMANDOS PARA ABREVIACIONES ====================
\newcommand{\ml}{Machine Learning}
\newcommand{\dl}{Deep Learning}
\newcommand{\ai}{Artificial Intelligence}
\newcommand{\api}{API}
\newcommand{\http}{HTTP}
\newcommand{\json}{JSON}

% ==================== COMANDOS PARA REFERENCIAS CRUZADAS ====================
% Referencia a figura
\newcommand{\figref}[1]{Figura~\ref{#1}}

% Referencia a tabla
\newcommand{\tabref}[1]{Tabla~\ref{#1}}

% Referencia a capítulo
\newcommand{\chapref}[1]{Capítulo~\ref{#1}}

% Referencia a sección
\newcommand{\secref}[1]{Sección~\ref{#1}}

% Referencia a ecuación (eqref ya existe en amsmath)
\newcommand{\Eqref}[1]{Ecuación~\eqref{#1}}

% Referencia a algoritmo
\newcommand{\Algref}[1]{Algoritmo~\ref{#1}}

% ==================== COMANDOS PARA ACRÓNIMOS ====================
\makeglossaries

% Definir acrónimos comunes
\newacronym{ml}{ML}{Machine Learning}
\newacronym{dl}{DL}{Deep Learning}
\newacronym{ai}{AI}{Inteligencia Artificial}
\newacronym{ann}{ANN}{Artificial Neural Network}
\newacronym{cnn}{CNN}{Convolutional Neural Network}
\newacronym{rnn}{RNN}{Recurrent Neural Network}
\newacronym{lstm}{LSTM}{Long Short-Term Memory}
\newacronym{api}{API}{Application Programming Interface}
\newacronym{rest}{REST}{Representational State Transfer}
\newacronym{http}{HTTP}{Hypertext Transfer Protocol}
\newacronym{json}{JSON}{JavaScript Object Notation}
\newacronym{sql}{SQL}{Structured Query Language}
\newacronym{nosql}{NoSQL}{Not Only SQL}
\newacronym{aws}{AWS}{Amazon Web Services}
\newacronym{saas}{SaaS}{Software as a Service}
\newacronym{roi}{ROI}{Return on Investment}
\newacronym{kpi}{KPI}{Key Performance Indicator}
\newacronym{pci}{PCI}{Payment Card Industry}
\newacronym{gdpr}{GDPR}{General Data Protection Regulation}
\newacronym{aml}{AML}{Anti-Money Laundering}
\newacronym{kyc}{KYC}{Know Your Customer}
\newacronym{rf}{RF}{Random Forest}
\newacronym{svm}{SVM}{Support Vector Machine}
\newacronym{knn}{KNN}{K-Nearest Neighbors}
\newacronym{xgboost}{XGBoost}{Extreme Gradient Boosting}

% ==================== COMANDOS PARA NOTAS Y COMENTARIOS ====================
% Nota del autor (para borrar antes de la versión final)
\newcommand{\nota}[1]{\textcolor{red}{\textbf{[NOTA: #1]}}}

% Pendiente
\newcommand{\pendiente}[1]{\textcolor{orange}{\textbf{[PENDIENTE: #1]}}}

% Revisar
\newcommand{\revisar}[1]{\textcolor{blue}{\textbf{[REVISAR: #1]}}}

% ==================== COMANDOS PARA ELEMENTOS ESPECIALES ====================
% Línea horizontal decorativa
\newcommand{\HRule}{\rule{\linewidth}{0.5mm}}

% Espacio vertical
\newcommand{\vsp}{\vspace{0.5cm}}

% Nueva página (solo si es necesario)
\newcommand{\nuevapagina}{\clearpage}

% ==================== COMANDOS PARA MÉTRICAS Y UNIDADES ====================
\newcommand{\accuracy}{\text{Accuracy}}
\newcommand{\precision}{\text{Precision}}
\newcommand{\recall}{\text{Recall}}
\newcommand{\fscore}{F_1\text{-score}}
\newcommand{\auc}{\text{AUC}}
\newcommand{\roc}{\text{ROC}}

% ==================== COMANDOS PARA DATASETS Y VARIABLES ====================
\newcommand{\dataset}{\mathcal{D}}
\newcommand{\trainset}{\mathcal{D}_{\text{train}}}
\newcommand{\testset}{\mathcal{D}_{\text{test}}}
\newcommand{\valset}{\mathcal{D}_{\text{val}}}

% ==================== COMANDOS ÚTILES PARA EDICIÓN ====================
% Comentario largo (no aparece en PDF)
\newcommand{\comentario}[1]{}

% Texto en desarrollo
\newcommand{\desarrollo}[1]{\textcolor{purple}{\textit{[En desarrollo: #1]}}}
